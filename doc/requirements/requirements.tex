%0       1         2         3         4         5         6         7         8
%2345678901234567890123456789012345678901234567890123456789012345678901234567890
%=======================================================================
\documentclass{article}
%=======================================================================


% INCLUDE DEVELOPMENT TEXT

\newcommand{\devel}[1]{\textbf{#1}}

% EXCLUDE DEVELOPMENT TEXT

% \newcommand{\devel}[1]{}


%=======================================================================
% Document layout
%=======================================================================

\setlength{\topmargin}{-0.5in}
\setlength{\oddsidemargin}{0.0in}
\setlength{\evensidemargin}{0.0in}
\setlength{\textwidth}{6.5in}
\setlength{\textheight}{9.0in}

%=======================================================================
% Packages
%=======================================================================
\newcounter{figctr}

\newcommand{\FIGURE}[3]{
\noindent
\parbox{\textwidth}{
%\ \\ \hrule \ \\
\begin{center}
#3
\end{center}%
\ \nolinebreak%
\refstepcounter{figctr}%
\begin{center}%
\begin{minipage}{7.0in}
\textbf{Figure \thefigctr}. #1
\end{minipage}
\end{center}
\label{#2}
%\ \\ \hrule \ \\
}}

\usepackage{wasysym}
\usepackage{epsfig}
\usepackage{url}

%=======================================================================
% Commands
%=======================================================================

\newcommand{\cello}{\textsf{Cello}}
\newcommand{\enzo}{\textsf{Enzo}}
\newcommand{\lcaperf}{\textsf{lcaperf}}
\newcommand{\lcatest}{\textsf{lcatest}}

\newcommand{\code}[1]{\textsf{#1}}
% not note because conflicts with beamer
\newcommand{\note}[1]{\devel{\eighthnote\ \textit{#1} \\}} 
\newcommand{\pargraph}[1]{\devel{\P\ \textbf{#1} \\}}

\newcommand{\todo}{\devel{$\circ$}}
\newcommand{\done}{\devel{$\bullet$}}
\newcommand{\useyes}{\textcolor{green}{$\bullet$}}
\newcommand{\usemaybe}{\textcolor{yellow}{$\bullet$}}
\newcommand{\useno}{\textcolor{red}{$\bullet$}}
\newcommand{\halfdone}{\devel{\textcolor{gray}{$\bullet$}}}

\newcommand{\TITLE}[3]{
\title{ {\huge #1}  \\ \vspace{0.1in}
     {\small Document Version: #3} \vspace{-0.1in}
    }
\author{      #2 \\
        Laboratory for Computational Astrophysics\\
        University of California, San Diego}
\maketitle}

%=======================================================================


%=======================================================================

\begin{document}

%=======================================================================
\TITLE{Software Requirements Specifications}{James Bordner}{\textbf{work in progress}}
%=======================================================================

%=======================================================================
\section{Introduction} \label{s:intro}
%=======================================================================

   \cello\ is a high-performance adaptive astrophysics application
   intended primarily to be run on massively hierarchical parallel
   supercomputers.  It combines leading edge software engineering
   techniques, parallel datastructures, parallelism techniques, and
   computational physics algorithms.

%-----------------------------------------------------------------------
\subsection{High-level Requirements}
%-----------------------------------------------------------------------

\begin{itemize}
\item Configurability
\item Adaptability
\item Flexibility
\item Performance
\item Scalability
\end{itemize}


%-----------------------------------------------------------------------
\subsubsection{Configurability}

   Configurability refers to the user-controlled component of
   mapping software and data-structures to the problem and hardware.
   It may be compile-time or run-time.

%-----------------------------------------------------------------------
\subsubsection{Adaptibility}

   Adaptivity refers to the software-controlled component of
   dynamically mapping data-structures to the problem and
   hardware.
 

   Adaptive meshes (SAMR and CAMR), to adapt data-structures to problem
   features in space.

   Adaptive time stepping, to adapt to varying timestepping
   requirements of the problem in space and time.

   Adaptive multi-algorithms, to automatically and dynamically choose
   the best of several algorithms to solve the problem.

   Adaptive data-structures, to automatically and dynamically choose
   the best of several data-structures to solve the problem.

   Adaptive parallelism, to make most efficient use of available
   hardware parallelism.



%-----------------------------------------------------------------------
\subsubsection{Flexibility}

%-----------------------------------------------------------------------
\subsubsection{Performance}

%-----------------------------------------------------------------------
\subsubsection{Scalability}

%=======================================================================
\section{Functional Requirements}
%=======================================================================

%-----------------------------------------------------------------------
\subsection{Physics}
%-----------------------------------------------------------------------

%-----------------------------------------------------------------------
\subsubsection{Hydrodynamics}

%-----------------------------------------------------------------------
\subsubsection{Self-gravity}

%-----------------------------------------------------------------------
\subsection{Data-Structures}
%-----------------------------------------------------------------------

%-----------------------------------------------------------------------
\subsubsection{Structured Adaptive Mesh Refinement}

%-----------------------------------------------------------------------
\subsubsection{Continuous Adaptive Mesh Refinement}

%-----------------------------------------------------------------------
\subsubsection{Particles}



%-----------------------------------------------------------------------
\subsection{Parallelism}
%-----------------------------------------------------------------------

MPI data parallel

MPI-2 One-sided data parallel

Functional parallelism

Task Definement and Scheduling

Multicore pipelining of physics components


Multilevel parallelism

Multiple parallelism paradigms (MPI + OpenMP)

Flexible parallelism paradigms: map parallelism to tasks

Automatic code generation (or other) of parallel tasks to implement
parallelism and optimize performance


%-----------------------------------------------------------------------
\subsection{Software Development}
%-----------------------------------------------------------------------


\end{document}

%==================================================================
