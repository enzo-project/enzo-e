%0       1         2         3         4         5         6         7         8
%2345678901234567890123456789012345678901234567890123456789012345678901234567890
%=======================================================================
\documentclass{article}
%=======================================================================


% INCLUDE DEVELOPMENT TEXT

\newcommand{\devel}[1]{\textbf{#1}}

% EXCLUDE DEVELOPMENT TEXT

% \newcommand{\devel}[1]{}


%=======================================================================
% Document layout
%=======================================================================

\setlength{\topmargin}{-0.5in}
\setlength{\oddsidemargin}{0.0in}
\setlength{\evensidemargin}{0.0in}
\setlength{\textwidth}{6.5in}
\setlength{\textheight}{9.0in}

%=======================================================================
% Packages
%=======================================================================
\newcounter{figctr}

\newcommand{\FIGURE}[3]{
\noindent
\parbox{\textwidth}{
%\ \\ \hrule \ \\
\begin{center}
#3
\end{center}%
\ \nolinebreak%
\refstepcounter{figctr}%
\begin{center}%
\begin{minipage}{7.0in}
\textbf{Figure \thefigctr}. #1
\end{minipage}
\end{center}
\label{#2}
%\ \\ \hrule \ \\
}}

\usepackage{wasysym}
\usepackage{epsfig}
\usepackage{url}

%=======================================================================
% Commands
%=======================================================================

\newcommand{\cello}{\textsf{Cello}}
\newcommand{\enzo}{\textsf{Enzo}}
\newcommand{\lcaperf}{\textsf{lcaperf}}
\newcommand{\lcatest}{\textsf{lcatest}}

\newcommand{\code}[1]{\textsf{#1}}
% not note because conflicts with beamer
\newcommand{\note}[1]{\devel{\eighthnote\ \textit{#1} \\}} 
\newcommand{\pargraph}[1]{\devel{\P\ \textbf{#1} \\}}

\newcommand{\todo}{\devel{$\circ$}}
\newcommand{\done}{\devel{$\bullet$}}
\newcommand{\useyes}{\textcolor{green}{$\bullet$}}
\newcommand{\usemaybe}{\textcolor{yellow}{$\bullet$}}
\newcommand{\useno}{\textcolor{red}{$\bullet$}}
\newcommand{\halfdone}{\devel{\textcolor{gray}{$\bullet$}}}

\newcommand{\TITLE}[3]{
\title{ {\huge #1}  \\ \vspace{0.1in}
     {\small Document Version: #3} \vspace{-0.1in}
    }
\author{      #2 \\
        Laboratory for Computational Astrophysics\\
        University of California, San Diego}
\maketitle}

%=======================================================================


%=======================================================================

\begin{document}

%=======================================================================
\title{ {\huge \enzo\ performance issues and solutions}  \\ \vspace{0.1in}
     \vspace{-0.1in} {\small Document Version: $Rev$ }}
\author{ James Bordner \\
        Laboratory for Computational Astrophysics\\
        University of California, San Diego}
\maketitle
%=======================================================================

%=======================================================================
\section{Introduction}
%=======================================================================

The purpose of this document is to assemble a list of issues of, and a
corresponding list of solutions to, \enzo's performance.  While
parallel scalability is the main performance issue, we will include
issues and solutions related to all aspects of serial and parallel
performance.  The goal is to provide a single reference for
performance issues related to \enzo, to help with the analysis and
prioritization of issues, and to aid the design and implementation of
solutions.  This will be used both for modifying the current \enzo\
application, as well as in the design and implementation of the
follow-on \cello\ application currently being developed.

%=======================================================================
\section{Application Performance}
%=======================================================================

Compared to those available today, upcoming generations of high
performance computing platforms are expected to be progressively
broader and deeper.  ``Broader'' in the sense of increased
computational parallelism, with the number of simultaneous
floating-point operations measured in the millions and more; and
``deeper'' in the sense that parallelism will be organized into ever
more layers (supernode, node, chip, core, FPU) and data will be
organized into deep memory hierarchies (archive, disk, main memory,
cache, registers) with ever increasing performance penalties between
levels.

For a scientific application to be truly efficient on high performance
computers, it must take the fullest advantage of both the machine's
breadth and depth.  Memory access patterns must make efficient use of
the memory hierarchy by maximizing data reuse in low levels by
increasing data locality, and hiding access latency in the higher
levels.  Communication within each parallel hierarchy level must be
efficiently managed.  Parallel computations must be evenly
load-balanced within each parallel hierarchy level.  High parallel
efficiency must be maintained at all levels, from nodes and supernodes
to cores and FPU's.  Global synchronization points must be minimized.
All software algorithms' computation, memory usage, and communication
must all be scalable with both problem size and machine size.
Software must be designed to efficiently map the software data
structures---even when they are dynamically adapting to changes in the
underlying scientific problem being solved---to the available hardware
resources.  And the software may need to adapt to changing hardware
resources, due to increased fault rates from increased hardware
complexity and size.  All of these and more are required
simultaneously of a parallel scientific application for it to be truly
efficient.

Fortuntely, new and improved tools are available for developers of
high performance scientific computing applications to take advantage
of the increasingly complex and growing hardware resources.  Improved
compiler parallelizing and optimization technology; new and improved
parallel algorithms and distributed data structures; better tools,
libraries and techniques for performance measurment and optimization;
new and improved libraries and languages for developing parallel
applications; continually improving parallel debugging tools; and
improved software engineering practices that help control the
ever-growing software complexity, and help developers write software
that is more reliable, flexible, reusable, modifyable, and
maintainable.

%=======================================================================
\section{\enzo\ performance issues}
%=======================================================================

\newcommand{\REF}[1]{\ref{#1}}
\newcommand{\TAG}[1]{\texttt{#1}}
\newcommand{\numgrids}{n}
\newcommand{\numprocs}{p}
\newcommand{\sizefield}{|F|}
\newcommand{\sizefields}{\sum_i^n |F_i|}
\newcommand{\sizegrid}{|G|}
\newcommand{\sizegridlocal}{|G_l|}
\newcommand{\sizegridremote}{|G_r|}


\begin{tabular}{c|l|l}
\REF{issue:amr-distribute} & \TAG{issue:amr-distribute} & AMR distributed data structure \\
\REF{issue:amr-neighbors} & \TAG{issue:amr-neighbors} & Sibling search \\
\REF{issue:amr-balance} & \TAG{issue:amr-balance} & Dynamic load balancing \\
\REF{issue:amr-rebuild} & \TAG{issue:amr-rebuild} & Rebuild hierarchy \\
\REF{issue:amr-ghost-update} & \TAG{issue:amr-ghost-update} & Boundary update \\
\REF{issue:particle-movement} & \TAG{issue:particle-movement} & Particle movement \\
\REF{issue:particle-distribution} & \TAG{issue:particle-distribution} & Particle distribution \\
\REF{issue:libraries} & \TAG{issue:libraries} & External libraries \\
\REF{issue:memory-fragment} & \TAG{issue:memory-fragment} & Memory fragmentation \\
\REF{issue:memory-hierarchy-use} & \TAG{issue:memory-hierarchy-use} & Hierarchical memory utilization \\
\REF{issue:method-gravity} & \TAG{issue:method-gravity} & Gravity solver \\
\REF{issue:method-timestep-global} & \TAG{issue:method-timestep-global} & Global timesteps \\
\REF{issue:method-timestep-sliver} & \TAG{issue:method-timestep-sliver} & Sliver timesteps \\
\REF{issue:fault-detect} & \TAG{issue:fault-detect} & Fault detection \\
\REF{issue:fault-recover} & \TAG{issue:fault-recover} & Fault recovery \\
\REF{issue:data-io} & \TAG{issue:data-io} & I/O performance and reliability \\
\REF{issue:data-analyse} & \TAG{issue:data-analyse} & Data analysis \\
\REF{issue:data-relocate} & \TAG{issue:data-relocate} & Data relocation \\
\REF{issue:data-archive} & \TAG{issue:data-archive} & Data archiving \\
\REF{issue:code-vectorize} & \TAG{issue:code-vectorize} & Vector utilization \\
\REF{issue:performance-measure} & \TAG{issue:performance-measure} & Evaluating modifications \\
\REF{issue:performance-tests} & \TAG{issue:performance-tests} & Test problems 
\end{tabular}

%-----------------------------------------------------------------------
\subsection{AMR distributed data structure} \label{issue:amr-distribute}
%-----------------------------------------------------------------------

\devel{Description of the issue}
\devel{Current implementation}
\devel{Current limitations}
\devel{Desired features}

 One scaling issue that we have determined must be addressed is how
 the AMR hierarchy data structure is stored.  Currently, the AMR
 hierarchy is represented using C++ \code{grid} objects, which are
 assigned to MPI processes.  Each \code{grid} may be conceptually
 either ``local'' or ``remote'', depending on whether the \code{grid}
 object is assigned to the local MPI process or not.

 Even though only ``local'' \code{grid} objects have storage allocated
 for data fields, each MPI process stores both local and remote
 \code{grid}s.  The total storage is thus \[ \numgrids (\numprocs |G|
 + |F|)\] where $\numgrids$ is the number of grids, $\numprocs$ is the
 number of MPI processes, $|G|$ is the size in bytes of a \code{grid}
 object excluding data fields, and $|F|$ is the size in bytes of the
 data fields in a \code{grid} object.  Currently, $|G|$ is $2248$.

 Since $|F| \gg |G|$, for moderate sized runs the memory requirement
 behaves as $\numgrids |F|$.  However, for large enough runs
 ($\numgrids$ and $\numprocs$ large), the memory for each processor
 storing all \code{grid}s begins to dominate the memory for data
 fields.  The overhead per processor is roughly $\numgrids |G|$.  This
 is independent of the number of processors used, so the maximum
 simulation size is currently limited by the amount of memory
 available for each processor, not the total number of processors
 available.

The main datastructure in \enzo-1.0.0 is a hierarchy of grids.  Each
grid is represented using a C++ \code{grid} object, and may be either
``local'' or ``remote'', depending on whether the
\code{grid} object is assigned to the processor or not (that is, if
\verb+gp->ProcessorNumber == MyProcessorNumber+, where
\code{gp} is a pointer to a \code{grid} object, \code{ProcessorNumber}
is the \code{grid} data member storing the MPI rank of the processor to
which the \code{grid} object is assigned, and \code{MyProcessorNumber}
is the MPI rank of the executing processor).

Even though only ``local'' \code{grid} objects have storage allocated
for data fields, all processors must store the member data for all
\code{grid} objects.  The total storage is thus 
\[ \numgrids (\numprocs |G| + |F|)\]
where $\numgrids$ is the number of grids, $\numprocs$ is the number of
processors, $|G|$ is the size in bytes of a
\code{grid} object~\footnote{\code{sizeof(grid)} = $129i + 16F + 3f +
133p$, where $i = $\code{sizeof(int)}, $F = $\code{sizeof(FLOAT)}, $f
= $\code{sizeof(float)}, and $p = $\code{sizeof(void *)}.  For 64-bit
everythings, this is 2248 bytes.}  excluding data fields, and $|F|$ is
the size in bytes of the data fields in a \code{grid} object.


Since $|F| \gg |G|$, for small runs the memory requirement behaves as
$n |F|$.  However, for large enough runs ($n$ and $p$ large), the
memory for each processor storing all \code{grid}s can begin to
dominate the memory for data fields.  The overhead per processor is
roughly $\numgrids |G|$, where $|G|$ is roughly $2K$ bytes.  This is
independent of the number of processors used, so the maximum
simulation size is limited by the amount of memory available for each
processor, not the total number of processors available.

%-----------------------------------------------------------------------
\subsection{Sibling search}\label{issue:amr-neighbors}
%-----------------------------------------------------------------------

\devel{Description of the issue}
\devel{Current implementation}
\devel{Current limitations}
\devel{Desired features}

Grid patches contain ghost zones which must be refreshed periodically
from neighboring grid patches.  This requires that each grid patch know
who its neighbors are.  Currently, this is done in \enzo\ using a
chaining mesh approach, which localizes the mesh and reduces the
number of grid-grid overlap checks.


%-----------------------------------------------------------------------
\subsection{Dynamic load balancing} \label{issue:amr-balance}
%-----------------------------------------------------------------------

\devel{Description of the issue}
\devel{Current implementation}
\devel{Current limitations}
\devel{Desired features}

As the AMR grid hierarchy adapts to evolving features in a simulation,
the workload can become unbalanced across processors.  This can limit
the parallel efficiency of \enzo, since less-heavily loaded processors
will be idle, while waiting for more heavily-loaded processors to
complete their computing tasks.

\devel{load balancing---current \enzo\ approach and limitations}
%
To even out the distribution of computation among processes, \enzo\
performs a simple dynamic load balancing step.  Currently, the
algorithm used is very simple: it relocates grid patches from heavily
loaded processors to lightly loaded processors, using a grid's memory
usage as an indicator of its computational load.  The algorithm has
many known disadvantages.  First, it does not take locality and
communication costs into accout, and second, the metric it uses for
estimating computational load can be inaccurate.

Want to improve by taking into account
\begin{itemize}
\item parent-child grid locality
\item sibling grid locality
\item costs of rebalancing 
\item communication versus computation tradeoff
\item adaptivity to hardware / software datastructure / physics evolution
\end{itemize}

%-----------------------------------------------------------------------
\subsection{Rebuild hierarchy} \label{issue:amr-rebuild}
%-----------------------------------------------------------------------

\devel{Description of the issue}
\devel{Current implementation}
\devel{Current limitations}
\devel{Desired features}



%-----------------------------------------------------------------------
\subsection{Boundary update} \label{issue:amr-ghost-update}
%-----------------------------------------------------------------------

\devel{Description of the issue}
\devel{Current implementation}
\devel{Current limitations}
\devel{Desired features}
\vspace{2in}

%-----------------------------------------------------------------------
\subsection{Particle movement} \label{issue:particle-movement}
%-----------------------------------------------------------------------

\devel{Description of the issue}
\devel{Current implementation}
\devel{Current limitations}
\devel{Desired features}
\vspace{2in}

%-----------------------------------------------------------------------
\subsection{Particle distribution} \label{issue:particle-distribution}
%-----------------------------------------------------------------------

\devel{Description of the issue}
\devel{Current implementation}
\devel{Current limitations}
\devel{Desired features}
\vspace{2in}

%-----------------------------------------------------------------------
\subsection{External libraries} \label{issue:libraries}
%-----------------------------------------------------------------------

\devel{Description of the issue}
\devel{Current implementation}
\devel{Current limitations}
\devel{Desired features}
\vspace{2in}

%-----------------------------------------------------------------------
\subsection{Memory fragmentation} \label{issue:memory-fragment}
%-----------------------------------------------------------------------

\devel{Description of the issue}
\devel{Current implementation}
\devel{Current limitations}
\devel{Desired features}
\vspace{2in}

%-----------------------------------------------------------------------
\subsection{Hierarchical memory utilization} \label{issue:memory-hierarchy-use}
%-----------------------------------------------------------------------

\devel{Description of the issue}
\devel{Current implementation}
\devel{Current limitations}
\devel{Desired features}
\vspace{2in}

%-----------------------------------------------------------------------
\subsection{Gravity solver} \label{issue:method-gravity}
%-----------------------------------------------------------------------

\devel{Description of the issue}
\devel{Current implementation}
\devel{Current limitations}
\devel{Desired features}

  \enzo\ often needs to efficiently solve Poisson's equation over the
  grid hierarchy, most notably to compute the gravitational
  acceleration at each point.  It does this using a combination FFT
  and multigrid approach, where an FFT is used on the root grid (to
  easily allow for periodic boundary conditions), and multigrid is
  used in each patch, with boundary conditions interpolated from
  parent to child.  This approach is quite fast (anecdotal evidence
  places it at least several times faster than the FLASH AMR code),
  but has two problems.  The first is that it produces a relatively
  noisy solution, which is generally fine for cosmological
  simulations, but more accurate results are required for the other
  applications that \enzo\  is now being used for.  The second problem is
  that the acceleration predicted in the ghost zones of a patch can be
  inconsistent with the solution derived in the sibling patch.  The
  solution across sibling grids can be matched with an iterative
  approach, but at the cost of more communication between patches. 

  We intend to switch gravity solvers and use the approach described
  in \cite{hg00}.  This is quite similar to our current approach but
  adds a correction cycle that uses the fine grid solution to improve
  the coarse grid result, which is then re-interpolated to the child
  grid.  This technique both dramatically decreases the size of the
  remaining error, and restores consistency between patches.  This
  will allow us to drop the current sibling iteration step and so
  reduce the amount of communication required.


%-----------------------------------------------------------------------
\subsection{Global timesteps} \label{issue:method-timestep-global}
%-----------------------------------------------------------------------

\devel{Description of the issue}
\devel{Current implementation}
\devel{Current limitations}
\devel{Desired features}

  Timesteps are determined globally: all grids on all processors
  within a level of the AMR hierarchy are advanced at the same
  timestep.  The issue is that, since the timestep is global, a global
  synchronization step must be performed.  For runs that use tens or
  hundreds of thousands of processors, this global synchronization can
  potentially become a bottleneck.  In principle, the timestep does
  not need to be determined globally, but could be determined on a
  grid by grid basis.  Not only would this not require a global
  synchronization, but it would permit grids to advance at larger
  timesteps than they otherwise would be restricted to.  A drawback
  would be how to handle the interface between neighboring grids that
  are being advanced at different timesteps.  It is not yet known how
  to do this without sacrificing accuracy, but is worthwhile to
  persue.

%-----------------------------------------------------------------------
\subsection{Sliver timesteps} \label{issue:method-timestep-sliver}
%-----------------------------------------------------------------------

\devel{Description of the issue}
\devel{Current implementation}
\devel{Current limitations}
\devel{Desired features}
  ``Sliver'' timesteps arise from global timestepping on successive
  levels of the hierarchy not being integral factors of one another.
  If the timestep on one level is $k$, and the timestep on the
  next-finer level is $k/2-\epsilon$, then three timesteps must be
  performed on the finer level instead of two.  It is not known how to
  handle these; however, one possibility would be to restrict
  timesteps between levels to be integral factors of each other,
  analagous to mesh refinement factor being an integer.  Another
  possibility would be to determine the finest-level timestep, then
  ``backtrack'' to coarser grids and adjusting their timesteps to
  avoid sliver timesteps.  It is not known whether this would be
  possible.

%-----------------------------------------------------------------------
\subsection{Fault detection} \label{issue:fault-detect}
%-----------------------------------------------------------------------

\devel{Description of the issue}
\devel{Current implementation}
\devel{Current limitations}
\devel{Desired features}
\vspace{2in}

%-----------------------------------------------------------------------
\subsection{Fault recovery} \label{issue:fault-recover}
%-----------------------------------------------------------------------

\devel{Description of the issue}
\devel{Current implementation}
\devel{Current limitations}
\devel{Desired features}
\vspace{2in}

%-----------------------------------------------------------------------
\subsection{I/O performance and reliability} \label{issue:data-io}
%-----------------------------------------------------------------------

\devel{Description of the issue}
\devel{Current implementation}
\devel{Current limitations}
\devel{Desired features}

``I/O speed and capacity will lag''

\vspace{2in}

%-----------------------------------------------------------------------
\subsection{Data analysis} \label{issue:data-analyse}
%-----------------------------------------------------------------------

\devel{Description of the issue}
\devel{Current implementation}
\devel{Current limitations}
\devel{Desired features}
   Off-site data movement

\vspace{2in}

%-----------------------------------------------------------------------
\subsection{Data relocation}\label{issue:data-relocate}
%-----------------------------------------------------------------------

\devel{Description of the issue}
\devel{Current implementation}
\devel{Current limitations}
\devel{Desired features}
\vspace{2in}

%-----------------------------------------------------------------------
\subsection{Data archiving}\label{issue:data-archive}
%-----------------------------------------------------------------------

\devel{Description of the issue}
\devel{Current implementation}
\devel{Current limitations}
\devel{Desired features}
\vspace{2in}

%-----------------------------------------------------------------------
\subsection{Vector utilization} \label{issue:code-vectorize}
%-----------------------------------------------------------------------

\devel{Description of the issue}
\devel{Current implementation}
\devel{Current limitations}
\devel{Desired features}
\vspace{2in}

%-----------------------------------------------------------------------
\subsection{Evaluating modifications}\label{issue:performance-measure}
%-----------------------------------------------------------------------

\devel{Description of the issue}
\devel{Current implementation}
\devel{Current limitations}
\devel{Desired features}
\vspace{2in}

%-----------------------------------------------------------------------
\subsection{Test problems} \label{issue:performance-tests}
%-----------------------------------------------------------------------

\devel{Description of the issue}
\devel{Current implementation}
\devel{Current limitations}
\devel{Desired features}
\vspace{2in}

%=======================================================================
\section{\enzo\ performance solutions}
%=======================================================================

\begin{tabular}{c|l|l}
\REF{solution:amr-grid-refactor} & \TAG{solution:amr-grid-refactor} & Grid class refactoring \\
\REF{solution:amr-cache} & \TAG{solution:amr-cache} & Cached AMR data structure  \\
\REF{solution:amr-large-grids} & \TAG{solution:amr-large-grids} & Favor larger grids  \\
\REF{solution:amr-dynamic-ghosts} & \TAG{solution:amr-dynamic-ghosts} & Dynamic ghost zones  \\
\REF{solution:amr-octree} & \TAG{solution:amr-octree} & Octree  \\
\REF{solution:amr-boxtree} & \TAG{solution:amr-boxtree} & Recursive Binary Box Tree (RBBT)  \\
\REF{solution:amr-grid-quantized} & \TAG{solution:amr-grid-quantized} & Quantized grid patch sizes  \\
\REF{solution:amr-balance-trees} & \TAG{solution:amr-balance-trees} & Load balance trees not patches  \\
\REF{solution:amr-balance-performance} & \TAG{solution:amr-balance-performance} & Load balance using performance measurements  \\
\REF{solution:amr-balance-hilbert} & \TAG{solution:amr-balance-hilbert} & Load balance using space-filling curves  \\
\REF{solution:amr-balance-hierarchical} & \TAG{solution:amr-balance-hierarchical} & Hierarchical load balancing  \\
\REF{solution:amr-traversal-local} & \TAG{solution:amr-traversal-local} & Traverse local grids only  \\
\REF{solution:amr-grid-reuse} & \TAG{solution:amr-grid-reuse} & Reuse existing grid classes  \\
\REF{solution:amr-balance-split} & \TAG{solution:amr-balance-split} & Inter-level refinement  \\
\REF{solution:particles-group} & \TAG{solution:particles-group} & Particle groups  \\
\REF{solution:parallel-hybrid} & \TAG{solution:parallel-hybrid} & Hybrid parallelism \\
\REF{solution:parallel-pgas} & \TAG{solution:parallel-pgas} & PGAS distributed AMR hierarchy  \\
\REF{solution:parallel-onesided} & \TAG{solution:parallel-onesided} & MPI-2 one-sided not MPI-1 two-sided  \\
\REF{solution:parallel-dynamic-procs} & \TAG{solution:parallel-dynamic-procs} & Dynamic process creation / deletion  \\
\REF{solution:parallel-dynamic-tasks} & \TAG{solution:parallel-dynamic-tasks} & Dynamic task allocation  \\
\REF{solution:parallel-data-analysis} & \TAG{solution:parallel-data-analysis} & Weakly-coupled parallel data analysis  \\
\REF{solution:parallel-subblocks} & \TAG{solution:parallel-subblocks} & Grid patch subblocks  \\
\REF{solution:method-p3dfft} & \TAG{solution:method-p3dfft} & P3DFFT for unigrid gravity \\
\REF{solution:method-hypre-fac} & \TAG{solution:method-hypre-fac} & HYPRE FAC for AMR gravity and radiation  \\
\REF{solution:performance-lcaperf} & \TAG{solution:performance-lcaperf} & \lcaperf\ performance monitoring  \\
\REF{solution:data-io-asynch} & \TAG{solution:data-io-asynch} & Asynchronous I/O  \\
\REF{solution:memory-management} & \TAG{solution:memory-management} & Large mallocs  
\end{tabular}

%-----------------------------------------------------------------------
\subsection{Grid class refactoring} \label{solution:amr-grid-refactor}
%-----------------------------------------------------------------------

For moderate numbers of processors, the current data structure used for
storing the AMR grid hierarchy is adequate.  Even though the hierarchy
topology (i.e. metadata) is stored redundantly on each processor, the
extra memory overhead involved is insignificant because the data
fields are vastly larger than \enzo's individual C++ \code{grid} objects.
However, as the number of processors increases, this memory overhead
increases as well.  For the processor counts required for
petascale-level computing, the storage overhead would dominate,
to the point where we would be memory rather than cpu limited.
Thus, for \enzo\ to scale to the petascale level, the memory overhead for
storing the AMR hierarchy must be reduced.

The memory required for storing \enzo's current AMR
data structure can be approximated as $\sizefield + \numgrids \numprocs
\sizegrid$, where the first term $\sizefield$ is the field
variable data, and the second term $\numgrids \numprocs \sizegrid$ is
the overhead for storing the grid hierarchy data structure.  Here
$\numgrids$ is the number of grid patches, $\numprocs$
is the number of processors, and $\sizegrid$ is the storage required
by a C++ \code{grid} object.

One approach to reducing the size of the overhead term would be
to split the \code{grid} class into two subclasses \code{grid\_local}
and \code{grid\_remote}, and use  \code{grid\_local} objects for local grids
that contain field data, and \code{grid\_remote} objects for grid
patches whose data fields reside on another processor.
This modification would change the overhead storage term from
$\numgrids \numprocs \sizegrid$ to $\numgrids ((\numprocs-1)
\sizegridremote + \sizegridlocal$), where $\sizegridremote$ is the
size of the \code{grid\_remote} class and $\sizegridlocal$ is the size
of the \code{grid\_local} class.  The advantage of doing this is that the
\code{grid\_remote} class would be made much smaller, since most of
the variables in the \code{grid} class are only required for local
grids.  Also, a vast majority of the grid classes are these much
smaller \code{grid\_remote} objects.  Thus the memory savings for this
modification would be quite large---a factor of $15$-$20$ over the
current approach.

 We have several plans for addressing this issue.  The first is to
 split the \code{grid} class into \code{grid\_local} and
 \code{grid\_remote} subclasses, and remove the unnecessary member
 data from the \code{grid\_remote} class.  Since most \code{grid}s (on
 average $(\numprocs-1)/\numprocs$) are \code{grid\_remote} grids, and
 since the bulk of the member data is only required for
 \code{grid\_local} grids, the memory savings for this modification
 will be very high: about a factor of $15$ to $20$.

%-----------------------------------------------------------------------
\subsection{Cached AMR data structure}  \label{solution:amr-cache}
%-----------------------------------------------------------------------
%-----------------------------------------------------------------------
\subsection{Favor larger grids}\label{solution:amr-large-grids}
%-----------------------------------------------------------------------

 A second improvement we are investigating is to modify the AMR
 hierarchy gridding algorithm to favor larger grid patches, and
 assigning multiple ($2$ to $8$) threads to each grid patch.  This
 would have several advantages.  First, it would increase scalability
 by allowing larger number of parallel tasks for a given AMR problem
 size.  Second, in would increase the volume-to-surface area ratio,
 decreasing inter-grid communication relative to computation, and
 reducing the ratio of ghost zones to computational zones.  Thirdly,
 the hierarchy would be smaller for a given problem size, which would
 reduce the AMR hierarchy management tasks, such as hierarchy
 rebuilding and load balancing.  Lastly, it could help reduce the
 frequency of hierarchy rebuilding, since larger grid patches would
 not need to be regridded as frequently as smaller ones.

%-----------------------------------------------------------------------
\subsection{Dynamic ghost zones} \label{solution:amr-dynamic-ghosts}
%-----------------------------------------------------------------------

  \enzo's PPM hydrodynamics algorithm requires a grid to have a level
  of ghost zones three deep to store data from neighboring and parent
  grids, which may lie on different processors.  The memory overhead
  required for storing these ghost zones can be very high, especially
  when the grid is small.  The ``half-efficiency'' point in terms of
  memory storage for cubical grid patch is $23^3$, meaning that a grid
  of that size has roughly the same number of active zones as ghost
  zones, and is higher for non-cubical grids.

  Currently, these ghost zones are stored permanently, allowing all
  communication to be done in a single step before the hydrodynamics
  solve.  However, since the ghost zones are only required in isolated
  locations in the code, and the values are always refreshed before
  the ghost zones are accessed, they do not need to be stored
  permanently.  The opposite extreme to storing ghost zones
  permanently would be to allocate them a grid patch at a time only
  when needed, and deallocated as soon as they are no longer required.
  In this case, the number of grids with ghost zones would be at most
  the number of active parallel threads.

  Although this would improve memory use significantly, it would
  unfortunately be extremely inefficient due to communication latency.
  Currently, with all ghost zones in all grids being refreshed
  together, this latency is hidden.

  However, it would be possible to optimize storage use versus hiding
  communication latency by partitioning grids into some small number
  of groups, and allocating ghost zones only a group at a time.  We
  propose to try this.

%-----------------------------------------------------------------------
\subsection{Octree} \label{solution:amr-octree}
%-----------------------------------------------------------------------

\vspace{2in}


%-----------------------------------------------------------------------
\subsection{Recursive Binary Box Tree (RBBT)}  \label{solution:amr-boxtree}
%-----------------------------------------------------------------------

\vspace{2in}


%-----------------------------------------------------------------------
\subsection{Quantized grid patch sizes} \label{solution:amr-grid-quantized}
%-----------------------------------------------------------------------

\vspace{2in}


%-----------------------------------------------------------------------
\subsection{Load balance trees not patches} \label{solution:amr-balance-trees}
%-----------------------------------------------------------------------

For example, subtrees of grids can be relocated instead of individual
grids, which would help maintain locality between nested grids.

%-----------------------------------------------------------------------
\subsection{Load balance using performance measurements}  \label{solution:amr-balance-performance}
%-----------------------------------------------------------------------

First, the estimate of computational load can be
greatly improved by using actual measured performance data from
\lcaperf\ instead of the current crude estimates.  This would be
particularly helpful in simulations where the cost of physics
algorithms are highly non-uniform, such as in the chemistry module,
which performs variable subcycling, or the hydrodynamics module in the
presence of strong isolated shocks.  

MPI + OpenMP model, tasks within a node could be balanced by assigning
grids to threads dynamically instead of statically when the grid patch
is created.  We also plan to incorporate ideas from the cutting-edge
dynamic load balancing research by Zhiling Lan, who has used \enzo\ to
develop load balancing algorithms that take into account heterogeneous
and fluctuating network performance, heterogeneous processors, and AMR
hierarchy features unique to cosmological simulations~\cite{LaTa06}.


%-----------------------------------------------------------------------
\subsection{Load balance using space-filling curves}  \label{solution:amr-balance-hilbert}
%-----------------------------------------------------------------------

The current load balancer allocates new patches on the same
processor as the parent patch and then moves them as required in an
attempt to retain parent-sibling locality.  However, performance
analysis shows that most communication is done to sibling grids, and
so we will develop an alternate load balancer which uses a space
filling curve (such as a Hilbert or morton curve) in order to both
load balance and maintain maximal sibling locality.  This technique
has been used successfully in other SAMR applications
\cite{WHH03,LWBH07}.

And
space-filling (e.g.~Hilbert or Morton) curves can be used to help
preserve locality between sibling grids.  For the flat MPI model,
space filling curves would also help reduce inter-node communication
by keeping more tightly-coupled grids on the same node.

%-----------------------------------------------------------------------
\subsection{Hierarchical load balancing}\label{solution:amr-balance-hierarchical}
%-----------------------------------------------------------------------

Finally, \enzo\ 's current load balancer does not address deep-memory
hierarchies, multi-core architectures and heterogeneous/hybrid design
that are common in the emerging HPC systems. On multi-core based
systems, the communication cost across nodes may be several orders of
magnitude higher than that within the shared memory nodes. To reduce
inter-node communication, the load balancer will allocate related
subgrids, e.g. parent and sibling subgrids, within the shared memory
nodes.  Within the current MPI implementation, this can be achieve
simply through the correct ordering of nodes within the space filling
curve described earlier.  Below, we will describe a more comprehensive
approach to the new hybrid architecture.

%-----------------------------------------------------------------------
\subsection{Traverse local grids only}\label{solution:amr-traversal-local}
%-----------------------------------------------------------------------

  \enzo's data structures are currently implemented such that a
  processor traverses a linked list of all grids in a level, tests
  each grid to see whether it is assigned to the given processor, and
  only perform computations on the grid data if the grid is assigned
  to the processor, otherwise it continues on to the next grid in the
  list.  For relatively small numbers of processors this is
  reasonable, but for very large numbers of processors, the overhead
  of traversing the entire global hierarchy is not scalable.

  An alternative would be to generate a second linked list containing
  only grids assigned to this processor.  This would eliminate the
  need for subsequent testing of whether a grid is assigned to the
  processor, and would also shorten the linked list by a factor of
  $P$, the number of processors.

%-----------------------------------------------------------------------
\subsection{Reuse existing grid classes} \label{solution:amr-grid-reuse}
%-----------------------------------------------------------------------

\vspace{2in}

%-----------------------------------------------------------------------
\subsection{Inter-level refinement} \label{solution:amr-balance-split}
%-----------------------------------------------------------------------

When load balancing two nested grids, instead of
sending child grid to another processor, split grids
in ``half'' to maintain parent-child locality.

%-----------------------------------------------------------------------
\subsection{Particle groups} \label{solution:particles-group}
%-----------------------------------------------------------------------

\vspace{2in}

%-----------------------------------------------------------------------
\subsection{Hybrid parallelism}\label{solution:parallel-hybrid}
%-----------------------------------------------------------------------

   ``basic hybrid will make stack replication worse''
   HYPRE can use OpenMP

  Switching from MPI to hybrid MPI + OpenMP could help \enzo's
  communication efficiency on hardware's hierarchical parallelism.
  Studies have shown that the overhead of OpenMP due to repeated
  thread spawning and false-sharing of cache lines can reduce
  performance.  Using OpenMP in a coarse-grain manner, by spawning
  threads just once at the beginning of a simulation, would control
  the parallel loop overhead.  While it's not known whether this
  hybrid approach would always be more efficient than stand-alone MPI,
  it will probably still be more efficient on some archituctures, or
  with some implementations of MP or OpenMP.  Assuming \enzo\ could
  still be run in MPI-only mode, adding the option to use MPI+OpenMP
  would only increase \enzo's efficiency.

  Another improvement would be to distribute the entire grid hierarchy
  on each processor.  The easiest approach would be to store one copy
  per shared memory node, which would decrease the memory storage
  further by a factor equal to the number of processors per node. This
  could easily be implemented with one of the hybrid parallel models
  described in the next section.

   While the above modifications to the AMR data structure should allow
   \enzo\ to run on machines with on the order of $10^4$ to $10^5$
   processors, extending to $10^6$ processors may require reducing the
   overhead even further.  The ultimate improvement memory-wise would
   be to store a single copy of the grid hierarchy, though depending
   on the node interconnect that would cause a communication
   bottleneck.  A refinement to this approach would be to store one
   copy of the hierarchy for every $M$ processors, where $M$ is some
   machine-dependent number chosen to balance the tradeoff between
   memory and communication overhead.

  The emergence of multi-core architectures will profoundly change
  parallel programming in high performance computing. Although the
  question of what is the best programming model for multicore
  architectures is largely unsettled, it is widely accepted that there
  are two common ways for parallel programming in multicore systems.
  One is \emph{the flat MPI model}, adopted by the current version of
  \enzo, in which separate single-threaded MPI processes are executed
  on each processing core. The other is called \emph{hybrid
  programming model}, where one MPI process is used per node and a
  multi-threaded, shared memory approach is exploited in the multiple
  cores. For example, OpenMP, Pthreads or the recently developed
  UPC/CAF may be used with MPI.  The disadvantage of the flat MPI
  model with hierarchical architectures is that it involves
  unnecessary overhead to copy data within a node.

  An additional factor of $8$ to $16$, depending on the number of
  computational elements used per shared memory node, can be gained by
  going to hybrid MPI + OpenMP parallel model.  Hybrid parallelism is
  known to be less efficient in general compared to using MPI alone,
  due to thread management overhead, cache coherence, etc.; however,
  we expect that the reduction in memory overhead for storing the AMR
  hierarchy will help performance as well as scalability, which will
  help counteract this probable loss of performance.  Furthermore,
  task assignment can easily be performed dynamically, so that the
  load balancing problem will only need to explicitly be done between
  nodes and not between each processing element.  Furthermore, since
  \enzo\ actually began as a shared memory parallel application before
  it was modified to use MPI, we do not expect adding OpenMP support
  to be too difficult a task. % Is this true?

%-----------------------------------------------------------------------
\subsection{PGAS distributed AMR hierarchy} \label{solution:parallel-pgas}
%-----------------------------------------------------------------------

 If the combined factor of $100$ to $300$ improvement that the above
 two modifications would provide is still insufficient to run large \enzo\
 AMR problems efficiently on $200K$ cores, we are also investigating
 additional ways to improve the AMR hierarchy data structure overhead.
 One modification we are evaluating is using UPC to store a single
 distributed copy of the AMR hierarchy topology rather than multiple
 redundant copies.  This would not just reduce the size of the
 overhead, but would indeed minimize it.  It would introduce some
 additional communication, but this could be managed by caching the
 portions of the hierarchy topology that are repeatedly accessed by
 the local MPI process.  

%-----------------------------------------------------------------------
\subsection{MPI-2 one-sided not MPI-1 two-sided}  \label{solution:parallel-onesided}
%-----------------------------------------------------------------------

  Switching from using MPI send/receives exclusively for all
  processor-to-processor communication to MPI2's one-sided
  communication would reduce synchronization overhead.  Additionally,
  it could reduce data movement and intermediate buffers, and would
  simplify the programming, since only one side of the transfer needs
  to be programmed instead of both.

%-----------------------------------------------------------------------
\subsection{Dynamic process creation / deletion} \label{solution:parallel-dynamic-procs}
%-----------------------------------------------------------------------

   For error fault tolerance

%-----------------------------------------------------------------------
\subsection{Dynamic task allocation} \label{solution:parallel-dynamic-tasks}
%-----------------------------------------------------------------------

\vspace{2in}

%-----------------------------------------------------------------------
\subsection{Weakly-coupled parallel data analysis}\label{solution:parallel-data-analysis}
%-----------------------------------------------------------------------

\vspace{2in}

%-----------------------------------------------------------------------
\subsection{Grid patch subblocks} \label{solution:parallel-subblocks}
%-----------------------------------------------------------------------

\vspace{2in}

%-----------------------------------------------------------------------
\subsection{P3DFFT for unigrid gravity} \label{solution:method-p3dfft}
%-----------------------------------------------------------------------

\vspace{2in}

%-----------------------------------------------------------------------
\subsection{HYPRE FAC for AMR gravity and radiation}  \label{solution:method-hypre-fac}
%-----------------------------------------------------------------------

\vspace{2in}

%-----------------------------------------------------------------------
\subsection{\lcaperf\ performance monitoring} \label{solution:performance-lcaperf}
%-----------------------------------------------------------------------

\lcaperf\ is a small yet practical toolkit
 designed to measure a wide range of performance-related data,
 including performance counter hardware data (via PAPI~\cite{BrDo00}),
 MPI communication (via MPI's profiling interface), dynamic memory
 allocation (via overloaded C++ \code{new} and \code{delete}
 operators), and software data structure-related metrics, e.g.~number
 of grids, number of grid zones, or number of timesteps per level (via
 user-defined counters), all for multiple user-defined code sections.
 \lcaperf's very flexible post-processing utilities in turn can
 generate almost arbitrary derived metrics, by applying arithmetic,
 finite-difference, quadrature, interpolation, and reduction
 operations to the collected performance data.  Since derived metrics
 can be defined in terms of multiple runs of the application, derived
 quantaties trivially encompass such metrics as load balance
 efficiency histories or parallel efficiency histories.  \lcaperf\ can
 in turn plot any derived metric versus any other derived metric via
 the \code{gnuplot} package.  \lcaperf\ is thus ideally suited to
 monitoring the comprehensive parallel performance of large-scale

 Our plan is to extend \lcaperf\ in three ways.  First, to support
 additional parallelism models, such as some subset of OpenMP, HPC, or
 COF, so that \lcaperf\ can more directly monitor the effectiveness of
 the proposed hybrid parallelization enhancements to \enzo.  Second,
 to add support in the API to allow the application to access its own
 collected performance data, which would enable \enzo\ to monitor its
 own performance, and hence adapt to it.  This functionality could be
 used immediately in the load-balancing module, which requires
 accurate performance estimates.  Third, to improve the useability of
 the post-processing utilities.  Currently the utilities are
 collectively powerful, but rather low-level and tedious for the user.
 We plan to provide higher-level utilites, controlled by user input
 files, that will ultimately allow the user or automatic regression
 testing suite to directly generate a comprehensive performance report
 tailored to the application for a collection of related runs.

  \note{Differences between parallel levels may require different
  parallel communication technology: processors in node share memory;
  cores within processors share memory bandwidth}

 To effectively improve \enzo's scalability, all aspects of its
 parallel performance must be measured, analysed, and evaluated.  To
 enable this, our group has written the \lcaperf\ performance toolkit
 specifically with \enzo's complex and dynamic data structures in
 mind.  \lcaperf\ can measure the full range of how \enzo\ uses a
 parallel platform.  This includes utilization of processors in terms
 of floating point operations and memory accesses (via
 PAPI~\cite{BrDo00}), utilization of the parallel interconnect in
 terms of MPI message counts and sizes (via MPI's PMPI profiling
 interface), dynamic memory usage in terms of current and high-water
 heap memory allocated (via overloaded C++ \code{new} and
 \code{delete} operators), and disk usage in terms of the amount of
 data read from and written to disk (via user-defined metrics
 associated with calls to HDF5).  

 All of these metrics are collected for each root-level cycle of the
 simulation, each processing phase within the cycle, each level of the
 AMR hierarchy, and each MPI process.  Additionally, information about
 dynamic data structures are recorded, including number of AMR grids
 assigned to each MPI process, the total number of grid zones on each
 MPI process, the number of timesteps in each AMR level, and (soon and
 trivially) the number of particles assigned to each MPI process.
 Lastly, \lcaperf's very flexible post-processing utilities can
 compute almost arbitrary derived metrics, by applying arithmetic,
 finite-difference, quadrature, interpolation, and reduction
 operations to the collected performance counters.  

 Since derived metrics can be defined in terms of multiple runs of the
 application, derived quantaties trivially encompass such metrics as
 load balance efficiency histories, parallel efficiency histories, and
 direct comparisons of all metrics between two code variations applied
 to the same problem.  Visualization is simple but effective:
 \lcaperf\ can plot any metric, recorded or derived, versus any other
 metric, via the open source \code{gnuplot} package.  \lcaperf\ is
 thus ideally suited to providing direct quantitative feedback about
 comprehensive parallel performance of \enzo\ as we implement our
 scaling improvements to bring \enzo\ AMR from the Terascale to the
 Petascale.

%-----------------------------------------------------------------------
\subsection{Asynchronous I/O} \label{solution:data-io-asynch}
%-----------------------------------------------------------------------

\vspace{2in}

%-----------------------------------------------------------------------
\subsection{Large mallocs} \label{solution:memory-management}
%-----------------------------------------------------------------------

\vspace{2in}

%==================================================================
\end{document}
%=======================================================================

