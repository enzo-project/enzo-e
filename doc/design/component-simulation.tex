%=======================================================================
\section{\code{Simulation} Component} \label{s:component-simulation}
%=======================================================================

Given \code{Physics}, \code{Methods}, and \code{Amr} or
\code{Particles} data structures, specifies and implements the
top-level sequencing and properties of the \code{Problem} (or
\code{Problems} in the case of an ensemble) to run.

\begin{itemize}
\item \code{Simulation}
\item \code{Ensemble}
\item \code{Control}
\item \code{Problem}
\item \code{Physics}
\item \code{Method}
\item \code{IO}
\end{itemize}

\subsection{Attributes.}

\subsection{Operations}

%=======================================================================
\subsection{\code{Ensemble} Component} \label{s:component-ensemble}
%=======================================================================

The \code{Ensemble} class is used to define a ensemble of simulations.  

\subsection{Attributes.}

\subsection{Operations}


%=======================================================================
\subsection{Control Component} \label{ss:component-control}
%=======================================================================


Global simulation control.

Output types and parameters

\begin{itemize}
\item checkpoint (dump all)
\item output (specific fields)
\item movies (type and rate)
\item analysis (type of analysis, rate)
\item level of output (files for timestep, time, etc.)
\end{itemize}

%-----------------------------------------------------------------------
\subsection{Use Cases}
%-----------------------------------------------------------------------
%-----------------------------------------------------------------------
\subsection{Parameters}
%-----------------------------------------------------------------------

