%=======================================================================
\section{Domain Component} \label{s:component-domain}
%=======================================================================

The \code{domain} function is used to specify properties of the
domain.  Domains are boxes aligned with the axes of the computational
coordinate system, and are uniquely determined by the spacial
dimension, and the lowest and highest points in the domain.

%-----------------------------------------------------------------------
\subsection{Use Cases}
%-----------------------------------------------------------------------

\begin{verbatim}
   domain { 
      dimension = 3
      lower     = <-3e9,-3e9,-3e9>
      upper     = <3e9,3e9,3e9>
   }
\end{verbatim}

\begin{verbatim}
   domain { 3, <-3e9,-3e9,-3e9>, <3e9,3e9,3e9> }  // Implicit ordering
\end{verbatim}

\begin{verbatim}
   domain { 
      dimension = 3
      upper     = 3e9        // expand scalar to vector
      lower     = -upper     // parameters can be accessed as values
   }
\end{verbatim}

Include errors.

%-----------------------------------------------------------------------
\subsection{Parameters}
%-----------------------------------------------------------------------

 \todo\ \textit{Decide: allow defaults?  allow optional parameters?  Special
 \code{OPT\_} prefix for optional parameters?  Write out explicit copy
 of input file?}

The \code{domain} function has three parameters

\begin{tabular}{lll} \\
Name & Type & Restrictions \\ \hline
\code{dimension} & Scalar & $1-3$ \\
\code{lower}     & Vector & length = \code{dimension} \\
\code{upper}     & Vector & length = \code{dimension}, \code{upper} $>$ \code{lower}
\end{tabular}

Lower and upper points are given in units given by \code{units},
described in \S\ref{s:units}.

%-----------------------------------------------------------------------
\subsection{Restrictions}
%-----------------------------------------------------------------------

\begin{enumerate}
\item The dimension must be 1, 2, or 3.
\item The number of coordinates in both lower and upper points must equal the dimension.
\item Each coordinate of the lower point must be strictly greater than the corresponding coordinate of the upper point.
\end{enumerate}


