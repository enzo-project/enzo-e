%=======================================================================
\section{Input file grammar} \label{s:input-grammar}
%=======================================================================

Input files can include other files using \code{\#include "}\textit{filename}\code{"}

%-----------------------------------------------------------------------
\subsection{Basic types}
%-----------------------------------------------------------------------

Parameters for a function are ordered.  Parameters may have default
values associated with them.  Parameters have names, which may be
included or not included.  Parameters can appear one on each line, or
multiple parameters per line separated by commas.

Input files can include other files using \code{\#include "}\textit{filename}\code{"}

Repeated text can be assigned to a macro using \code{\#declare "}\textit{macro}\code{ = }\textit{function}


\begin{tabular}{|l|l|}\hline
\textbf{Type} & \textbf{Example} \\ \hline
section     & \code{Region \{ \ldots \} } \\
string      & \code{"a string"} \\
scalar      & \code{"1.6e42"} \\
list        & \code{["a list",X,1,2]} \\
constant    & \code{X},\code{Y},\code{Z}, etc. \\
function    & \code{sin(2*x)} \\
expression  & \code{X + Y * sin(2*X / PI)} \\
parameter   & \code{Hydro:method} \\
assignment  & \code{t := T + Y} \\
inequality  & \code{X + Y $<$ 0.5} \\
sets        & \code{R = r1 + r2$\setminus$r3} \\ \hline
\end{tabular}

%-----------------------------------------------------------------------
\subsubsection{Sections}
%-----------------------------------------------------------------------

A \textit{section} is a named region enclosed by braces, e.g.

\begin{verbatim}
Region { 
   \textit{parameter declarations} 
}
\end{verbatim}


Parameters are typed, and may have default values associated with
them.  Parameters can appear one on each line, or multiple parameters
per line separated by semicolons.

%-----------------------------------------------------------------------
\subsubsection{Strings}

Strings are enclosed in double-quotes `\code{"}'.  Example: \code{"density"}

%-----------------------------------------------------------------------
\subsubsection{Scalars}

Scalars are represented as C floating-point numbers.  Example: 3.0e-19.

%-----------------------------------------------------------------------
\subsubsection{Lists} 

Lists are enclosed in square-brackets `\code{$[$}' and `\code{$]$}'.  Example: \code{$[$3e9,3e9,3e9$]$}.

%-----------------------------------------------------------------------
\subsubsection{Constants} 

There are constants defined that can be accessed in the input file.  They
may not actually be ``constant'', such as the x-axis coordinate
in problem units ``\code{x}''.

\begin{tabbing}
xxx\=xxxxxxx\=xxxxxxxxxxxxxxxxxxxxxxxxxxxxxxxx\=\kill
\> \code{x},\code{y},\code{z} \> Coordinates in problem distance units. \\
\> \code{t} \> Time in problem time units. \\
\> \code{X},\code{Y},\code{Z} \> Coordinates in computational units. \\
\> \code{T} \> Time in computational units. \\
\> \code{PI} \> $\pi$ \\
\end{tabbing}


%-----------------------------------------------------------------------
\subsubsection{Function} 

\code{abs()},
\code{min()},
\code{max()},
\code{sqrt()},
\code{exp()},
\code{avg()},
\code{log()}


%-----------------------------------------------------------------------
\subsubsection{Expressions} 

$x + y$,
$x - y$,
$x * y$,
$x / y$

%-----------------------------------------------------------------------
\subsubsection{Parameters} 
%-----------------------------------------------------------------------
\subsubsection{Assignment} 
%-----------------------------------------------------------------------
\subsubsection{Inequality} 

$x < y$
$x <= y$
$x > y$
$x >= y$
$A \&\& B$
$A || B$
$A == B$
$A != B$

\subsubsection{Sets}

$R * S$
$R + S$
$R - S$
$-S$



