%0       1         2         3         4         5         6         7         8
%2345678901234567890123456789012345678901234567890123456789012345678901234567890

\documentclass[11pt]{article}


% INCLUDE DEVELOPMENT TEXT

\newcommand{\devel}[1]{\textbf{#1}}

% EXCLUDE DEVELOPMENT TEXT

% \newcommand{\devel}[1]{}


%=======================================================================
% Document layout
%=======================================================================

\setlength{\topmargin}{-0.5in}
\setlength{\oddsidemargin}{0.0in}
\setlength{\evensidemargin}{0.0in}
\setlength{\textwidth}{6.5in}
\setlength{\textheight}{9.0in}

%=======================================================================
% Packages
%=======================================================================
\newcounter{figctr}

\newcommand{\FIGURE}[3]{
\noindent
\parbox{\textwidth}{
%\ \\ \hrule \ \\
\begin{center}
#3
\end{center}%
\ \nolinebreak%
\refstepcounter{figctr}%
\begin{center}%
\begin{minipage}{7.0in}
\textbf{Figure \thefigctr}. #1
\end{minipage}
\end{center}
\label{#2}
%\ \\ \hrule \ \\
}}

\usepackage{wasysym}
\usepackage{epsfig}
\usepackage{url}

%=======================================================================
% Commands
%=======================================================================

\newcommand{\cello}{\textsf{Cello}}
\newcommand{\enzo}{\textsf{Enzo}}
\newcommand{\lcaperf}{\textsf{lcaperf}}
\newcommand{\lcatest}{\textsf{lcatest}}

\newcommand{\code}[1]{\textsf{#1}}
% not note because conflicts with beamer
\newcommand{\note}[1]{\devel{\eighthnote\ \textit{#1} \\}} 
\newcommand{\pargraph}[1]{\devel{\P\ \textbf{#1} \\}}

\newcommand{\todo}{\devel{$\circ$}}
\newcommand{\done}{\devel{$\bullet$}}
\newcommand{\useyes}{\textcolor{green}{$\bullet$}}
\newcommand{\usemaybe}{\textcolor{yellow}{$\bullet$}}
\newcommand{\useno}{\textcolor{red}{$\bullet$}}
\newcommand{\halfdone}{\devel{\textcolor{gray}{$\bullet$}}}

\newcommand{\TITLE}[3]{
\title{ {\huge #1}  \\ \vspace{0.1in}
     {\small Document Version: #3} \vspace{-0.1in}
    }
\author{      #2 \\
        Laboratory for Computational Astrophysics\\
        University of California, San Diego}
\maketitle}

%=======================================================================


\begin{document}

%==================================================================
\TITLE{\amrSolve \\ Software Requirements Specifications}{James Bordner}{0.94}
%==================================================================

%========================================================================
\section{Modification History}
%========================================================================

\BeginDESCRIPTION
\item[0.94 (2004-04-01):] Added definition of neighbors
\item[0.93 (2004-03-18):] Added Fields; revised content
\item[0.92 (2004-03-17):] Added factor of four to grid resolution
\item[0.91 (2004-03-16):] Revised content
\item[0.91 (2004-03-16):] Added Modification History section
\EndDESCRIPTION

%==================================================================
\section{Introduction}
%==================================================================

\amrSolve\ will be a linear solver package for patch-based AMR applications.
The basic objects involved are 
%
\class{Grid}s (\S\ref{ss:grid}), 
\class{Level}s (\S\ref{ss:level}), 
\class{Hierarchy}s (\S\ref{ss:hierarchy}), 
\class{Field}s (\S\ref{ss:field}), 
\class{Vector}s (\S\ref{ss:vector}), 
\class{Matrix}es (\S\ref{ss:matrix}), and
\class{Solver}s (\S\ref{ss:solver}).

%==================================================================
\section{Object specifications}
%==================================================================


%==================================================================
\subsection{\class{Grid}s} \label{ss:grid}
%==================================================================

\class{Grid}s implicitly represent a set of vertices arranged in
a grid pattern in a $3$D box in space.

\BeginENUMERATE
\item \class{Grid}s are $3$D, but can be interpreted as $1$D or $2$D if some axis lengths are one.
\item A \class{Grid} has a position in physical space, determined by any two opposing corners.
\item A \class{Grid} has a size in computational space, determined by the number of vertices along each axis.
\item Spacing between vertices along each axis is uniform.
\item \class{Grid}s do not contain data (see \S\ref{ss:field} on \class{Field}s).
\item Each \class{Grid} is  classified as either a \textit{local} \class{Grid} or a \textit{remote} \class{Grid}.
\item Each \class{Grid} is local on exactly one processor.
\item All non-root level \class{Grid}s have a unique \textit{parent} \class{Grid} in
    the next-coarser \class{Level}.
\item A \class{Grid} has a set of adjacent \textit{neighbor} \class{Grid}s.
  \class{Grid}s are adjacent iff the ghost zone of one \class{Grid} is a real
    zone in the other \class{Grid}. (This is a symmetric relationship).
\item A \class{Grid} has a set of \textit{child} \class{Grid}s in
the next-finer \class{Level}.
\item Corner vertices of a \class{Grid} must coincide with
vertices of its parent \class{Grid}.
\EndENUMERATE


%==================================================================
\subsection{\class{Level}s} \label{ss:level}
%==================================================================

\class{Level}s encapsulate all grids within a single level of a 
\class{Hierarchy}.

\BeginENUMERATE
\item A \class{Level} is an unordered list of \class{Grid}s
\item \class{Grid}s within a \class{Level} must be disjoint.
\item \class{Grid}s within a \class{Level} have the same spacing between vertices
\EndENUMERATE

%==================================================================
\subsection{\class{Hierarchy}s} \label{ss:hierarchy}
%==================================================================

A \class{Hierarchy} represents an entire AMR hierarchy.

\BeginENUMERATE
\item A \class{Hierarchy} is an ordered list of \class{Level}s
\item \class{Grid}s have at most one parent
\item Resolution changes by a factor of two or four between \class{Grid} \class{Level}s
\item Corner vertices of a non-root \class{Grid} are aligned with vertices in the \class{Grid}'s parent
\item \class{Grid}s in non-successive \class{Level}s cannot be adjacent
\item The union of \class{Grid}s in the root \class{Level} form a box
\EndENUMERATE

%==================================================================
\subsection{Fields} \label{ss:field}
%==================================================================

The following are assumptions related to fields on a \class{Hierarchy}

\BeginENUMERATE
\item Fields map \class{Grid}s to arrays or to \texttt{NULL}
\item An array for a \class{Grid} exists on exactly one processor, depending
  on if the \class{Grid} is \textit{local} or \textit{remote}.
\item An array may have a \textit{ghost} layer with a depth of at most $1$.
\EndENUMERATE

%==================================================================
\subsection{\class{Vector}s} \label{ss:vector}
%==================================================================

\BeginENUMERATE
\item \class{Vector}s can be defined on a single \class{Grid} (\textit{grid}-\class{Vector}), \class{Level} (\textit{level}-\class{Vector}), or on an entire \class{Hierarchy} (\textit{hierarchy}-\class{Vector}).
\item A \class{Vector} is composed of a \class{Field}, and either a \class{Hierarchy}, \class{Level}, or \class{Grid}.
\item A \class{Vector}'s elements are associated with either \class{Grid} vertices (\textit{nodal}-\class{Vector}) or \class{Grid} zone centers (\textit{zone}-\class{Vector})
\item A \class{Vector} may have elements covered by finer-level \class{Grid}s
\item A \class{Vector} may have a ghost vertex layer of depth between $0$ and $1$ 
\EndENUMERATE

%==================================================================
\subsection{\class{Matrix}es}  \label{ss:matrix}
%==================================================================

A \class{Matrix} represents a linear operator defined on \class{Vector}s in an AMR \class{Hierarchy}.

\BeginENUMERATE
\item A \class{Matrix} can be applied to a \class{Vector} ($y \leftarrow A x$)
\item $x$ and $y$ can be both \class{Hierarchy} \class{Vector}s
\item $x$ and $y$ can be level-\class{Vector}s at most one level apart
\item $x$ and $y$ can be both nodal-\class{Vector}s, both zone-\class{Vector}s, or one of each
\item \class{Matrix}es must support $7$-point and $19$-point stencil operators
\item \class{Matrix}es must support specialized discretizations along \class{Grid} faces, edges, and corners
\EndENUMERATE

%==================================================================
\subsection{\class{Solver}s} \label{ss:solver}
%==================================================================

\class{Solver}s represents linear solvers.

\BeginENUMERATE
\item  A  \class{Solver} must exist to solve the Poisson equation
\item  A  \class{Solver} must exist to solve linear equations arising from the FLD method for radiation transport
\item  A  \class{Solver} must exist to solve linear equations arising from the FTEF method for radiation transport
\item  A  \class{Solver} must exist to solve linear equations arising from the ``Paschos'' method for radiation transport
\item Must support periodic, dirichlet, and Neumann boundary conditions
\EndENUMERATE


%==================================================================
\end{document}
%==================================================================


