%0       1         2         3         4         5         6         7         8
%2345678901234567890123456789012345678901234567890123456789012345678901234567890

\documentclass[11pt]{article}

% \newcommand{\includepsgrid}{\psgrid}
\newcommand{\includepsgrid}{}


% INCLUDE DEVELOPMENT TEXT

\newcommand{\devel}[1]{\textbf{#1}}

% EXCLUDE DEVELOPMENT TEXT

% \newcommand{\devel}[1]{}


%=======================================================================
% Document layout
%=======================================================================

\setlength{\topmargin}{-0.5in}
\setlength{\oddsidemargin}{0.0in}
\setlength{\evensidemargin}{0.0in}
\setlength{\textwidth}{6.5in}
\setlength{\textheight}{9.0in}

%=======================================================================
% Packages
%=======================================================================
\newcounter{figctr}

\newcommand{\FIGURE}[3]{
\noindent
\parbox{\textwidth}{
%\ \\ \hrule \ \\
\begin{center}
#3
\end{center}%
\ \nolinebreak%
\refstepcounter{figctr}%
\begin{center}%
\begin{minipage}{7.0in}
\textbf{Figure \thefigctr}. #1
\end{minipage}
\end{center}
\label{#2}
%\ \\ \hrule \ \\
}}

\usepackage{wasysym}
\usepackage{epsfig}
\usepackage{url}

%=======================================================================
% Commands
%=======================================================================

\newcommand{\cello}{\textsf{Cello}}
\newcommand{\enzo}{\textsf{Enzo}}
\newcommand{\lcaperf}{\textsf{lcaperf}}
\newcommand{\lcatest}{\textsf{lcatest}}

\newcommand{\code}[1]{\textsf{#1}}
% not note because conflicts with beamer
\newcommand{\note}[1]{\devel{\eighthnote\ \textit{#1} \\}} 
\newcommand{\pargraph}[1]{\devel{\P\ \textbf{#1} \\}}

\newcommand{\todo}{\devel{$\circ$}}
\newcommand{\done}{\devel{$\bullet$}}
\newcommand{\useyes}{\textcolor{green}{$\bullet$}}
\newcommand{\usemaybe}{\textcolor{yellow}{$\bullet$}}
\newcommand{\useno}{\textcolor{red}{$\bullet$}}
\newcommand{\halfdone}{\devel{\textcolor{gray}{$\bullet$}}}

\newcommand{\TITLE}[3]{
\title{ {\huge #1}  \\ \vspace{0.1in}
     {\small Document Version: #3} \vspace{-0.1in}
    }
\author{      #2 \\
        Laboratory for Computational Astrophysics\\
        University of California, San Diego}
\maketitle}

%=======================================================================


\begin{document}

%\newcommand{\includepsgrid}{\psgrid}
% \newcommand{\includepsgrid}{}

%=======================================================================
% Point
%=======================================================================

   \newcommand{\drawClassPoint}{%
      \umlClass[umlShadow=false]{Point}{%
         \hline
         Point () \\
         get () \\
         set () \\
	 operator () () \\
         operator <= () \\
         sort ()%
   }}

   \newcommand{\umlPoint}{
   \begin{center}
   \begin{pspicture}(3,3.5)\includepsgrid
     \rput[lb](0,0){\rnode{ClassPoint}{\drawClassPoint}}
   \end{pspicture}
   \end{center}}

%=======================================================================
% Grid
%=======================================================================

   \newcommand{\drawClassGrid}{%
      \umlClass[umlShadow=false]{Grid}{%
         \hline
         + Grid () \\
         + \textit{attach} () \\
         + \textit{detach} () \\
         + size() \\
         + range() \\
         + center() \\
         + isNeighbor() \\
         + map () \\
         + invmap () \\
         + numNeighbors() \\
         + numChildren() \\
         + neighbor() \\
         + child() \\
         + parent() \\
         + volume () \\
         + level () \\
         + hierarchy () \\
         + summary () %
   }}

   \newcommand{\umlGrid}{
   \begin{center}
   \begin{pspicture}(4,9)\includepsgrid
     \rput[lb](0,0){\rnode{ClassGrid}{\drawClassGrid}}
   \end{pspicture}
   \end{center}}

%=======================================================================
% Level
%=======================================================================

   \newcommand{\drawClassLevel}{%
      \umlClass[umlShadow=false]{Level}{%
         \hline
         + Level () \\
         + \textit{attach} () \\
         + \textit{detach} () \\
         + numGrids() \\
         + grid() \\
         + volume () \\
         + covering() \\
         + finer() \\
         + coarser() \\
         + hierarchy() \\
         - assertNeighbors() %
   }}

   \newcommand{\umlLevel}{
   \begin{center}
   \begin{pspicture}(4,6)\includepsgrid
     \rput[lb](0,0){\rnode{ClassLevel}{\drawClassLevel}}
   \end{pspicture}
   \end{center}}

%=======================================================================
% Hierarchy
%=======================================================================

   \newcommand{\drawClassHierarchy}{%
      \umlClass[umlShadow=false]{Hierarchy}{%
         \hline
         + Hierarchy () \\
         + \textit{attach} () \\
         + \textit{detach} () \\
         + numLevels() \\
         + numGrids() \\
         + level() \\
         - assertParent() %
   }}

   \newcommand{\umlHierarchy}{
   \begin{center}
   \begin{pspicture}(4,4)\includepsgrid
     \rput[lb](0,0){\rnode{ClassHierarchy}{\drawClassHierarchy}}
   \end{pspicture}
   \end{center}}

%=======================================================================
% Iterator
%=======================================================================

   \newcommand{\drawClassIterator}{%
      \umlClass[umlShadow=false]{\textit{Iterator}}{%
         \hline
         Iterator () \\
         operator ++() \\
         operator * () \\
         operator begin () \\
         operator end () \\
         operator reset ()%
   }}

   \newcommand{\drawClassItGrids}{%
      \umlClass[umlShadow=false]{ItGrids}{%
   }}

   \newcommand{\drawClassItLevels}{%
      \umlClass[umlShadow=false]{ItLevels}{%
   }}

   \newcommand{\drawClassItChildren}{%
      \umlClass[umlShadow=false]{ItChildren}{%
   }}

   \newcommand{\drawClassItNeighbors}{%
      \umlClass[umlShadow=false]{ItNeighbors}{%
   }}

   \newcommand{\umlIterator}{%
   \begin{center}
   \begin{pspicture}(8,3.5)\includepsgrid
     \rput[lb](0,0){\rnode{ClassIterator}{\drawClassIterator}}
     \rput[lb](5.5,3){\rnode{ClassItLevels}{\drawClassItLevels}}
     \rput[lb](5.5,2){\rnode{ClassItGrids}{\drawClassItGrids}}
     \rput[lb](5.5,1){\rnode{ClassItChildren}{\drawClassItChildren}}
     \rput[lb](5.5,0){\rnode{ClassItNeighbors}{\drawClassItNeighbors}}
     \pnode(4,0.5){pnode1}
   \end{pspicture}
   \ncEVE[armA=1]{ClassIterator}{ClassItLevels}
   \ncEVE[armA=1]{ClassIterator}{ClassItGrids}
   \ncEVE[armA=1]{ClassIterator}{ClassItChildren}
   \ncEVE[armA=1]{ClassIterator}{ClassItNeighbors}
   \ncputicon{umlHerit}
   \end{center}}


%=======================================================================
% Vector
%=======================================================================

   \newcommand{\drawClassVectorAbc}{%
      \umlClass[umlShadow=false]{\textit{VectorAbc}}{%
         \hline
         + VectorAbc () \\
         + \textit{attach}() \\
         + \textit{detach}() \\
         + dot() \\
         + zaxpy() \\
         + set() \\
         + scale() \\
         + copy() \\
         + refresh()%
   }}

   \newcommand{\drawClassVectorGrid}{%
      \umlClass[umlShadow=false]{VectorGrid}{%
   }}

   \newcommand{\drawClassVectorLevel}{%
      \umlClass[umlShadow=false]{VectorLevel}{%
   }}

   \newcommand{\drawClassVectorHierarchy}{%
      \umlClass[umlShadow=false]{VectorHierarchy}{%
   }}

   \newcommand{\umlVector}{
   \begin{center}
   \begin{pspicture}(7.5,4.5)\includepsgrid
     \rput[lb](0,0){\rnode{ClassVectorAbc}{\drawClassVectorAbc}}
     \rput[lb](5,3){\rnode{ClassVectorGrid}{\drawClassVectorGrid}}
     \rput[lb](5,2){\rnode{ClassVectorLevel}{\drawClassVectorLevel}}
     \rput[lb](5,1){\rnode{ClassVectorHierarchy}{\drawClassVectorHierarchy}}
     \pnode(4.5,0.5){pnode1}
   \end{pspicture}
   \ncEVE[armA=1]{ClassVectorAbc}{ClassVectorGrid}
   \ncEVE[armA=1]{ClassVectorAbc}{ClassVectorLevel}
   \ncEVE[armA=1]{ClassVectorAbc}{ClassVectorHierarchy}
   \ncputicon{umlHerit}
   \end{center}}

%=======================================================================
% Matrix
%=======================================================================

   \newcommand{\drawClassMatrix}{%
      \umlClass[umlShadow=false]{Matrix}{%
         \hline
         Matrix () \\
         size() \\
         matvec() \\
         residual()%
   }}

   \newcommand{\umlMatrix}{
   \begin{center}
   \begin{pspicture}(2.5,2.5)\includepsgrid
     \rput[lb](0,0){\rnode{ClassMatrix}{\drawClassMatrix}}
   \end{pspicture}
   \end{center}}

%=======================================================================
% Solver
%=======================================================================

   \newcommand{\drawClassSolver}{%
      \umlClass[umlShadow=false]{Solver}{%
         \hline
         Solver () \\
         apply() \\
         status()%
   }}

   \newcommand{\umlSolver}{
   \begin{center}
   \begin{pspicture}(2,2)\includepsgrid
     \rput[lb](0,0){\rnode{ClassSolver}{\drawClassSolver}}
   \end{pspicture}
   \end{center}}


%========================================================================
\TITLE{\amrSolve \\ Software Design Description}{James Bordner}{0.1.0}
%========================================================================

%\tableofcontents
%========================================================================
% \section{Modification History}
%========================================================================

% \BeginDESCRIPTION
% \item[0.972 (2004-04-02):] Created Geomview class
% \item[0.971 (2004-04-02):] Modified geomview() return
% \item[0.97 (2004-04-01):] Added Grid::isNeighbor()
% \item[0.97 (2004-04-01):] Added Grid::, Level::, Hierarchy::geomview()
% \item[0.96 (2004-03-29):] Revised \class{Grid}/\class{Level}/\class{Hierarchy} design
% \item[0.95 (2004-03-23):] Revised up to \class{Iterator}s
% \item[0.95 (2004-03-23):] Added Modification History section
% \EndDESCRIPTION

%========================================================================
\section{Introduction}
%========================================================================

This document serves to specify the design of \amrSolve. 

%------------------------------------------------------------------------
\FIGURE{\amrSolve\ Package}{f:uml-amrSolve}{
\begin{center}
\epsfig{file=uml-amrSolve.\GRAPHICSSUFFIX,width=1.5in}
\end{center}}
%------------------------------------------------------------------------

\amrSolve\ classes are described next in \S\ref{s:amrsolve}: 
\class{Point} in \S\ref{ss:point}, 
\class{Grid} in \S\ref{ss:grid}, 
\class{Level} in \S\ref{ss:level}, 
\class{Hierarchy} in \S\ref{ss:hierarchy}, 
\class{Iterator} in \S\ref{ss:iterator}, 
% \class{Field} in \S\ref{ss:field}, 
\class{Vector} in \S\ref{ss:vector}, 
\class{Matrix} in \S\ref{ss:matrix}, 
\class{Stencil} in \S\ref{ss:stencil}, 
\class{Discret} in \S\ref{ss:discret},
\class{Solver} in \S\ref{ss:solver},
\class{Precon} in \S\ref{ss:precon}, and
\class{Geomview} in \S\ref{ss:geomview}.

%========================================================================
\section{Classes} \label{s:amrsolve}
%========================================================================


%------------------------------------------------------------------------
\subsection{\class{Point} Class} \label{ss:point}
%------------------------------------------------------------------------

   A \class{Point} is simply a position in $3$-space.  Units are
   determined by the calling application.  Coordinates are stored as a
   ``\code{Coordinate}'', which is simply a \code{typedef} to a
   \code{double} by default, which is defined in \code{Scalar.h}.

%**********************************************************************
   \umlPoint
%**********************************************************************

%-----------------------------------------------------------------------
\subsubsection{\class{Point} Functions}
%-----------------------------------------------------------------------

    \functionbf{$+$ void Point::Point ()} \\
    Constructor for the \class{Point} class.  Initializes to $(0,0,0)$.

    \functionbf{$+$ void Point::get (Coordinate \& x1, Coordinate \& x2, Coordinate \& x3) const} \\
    Function for getting a \class{Point}'s coordinates.

    \functionbf{$+$ void Point::set (Coordinate x1, Coordinate x2, Coordinate x3)} \\
    Function for setting a \class{Point}'s coordinates.

    \functionbf{$+$ Coordinate Point::operator () (int i) const} \\ Function for reading
    the \class{Point} position along the given coordinate axis.

    \functionbf{$+$ Coordinate \& Point::operator () (int i)} \\ Function for
    writing the \class{Point} position along the given coordinate
    axis.

    \functionbf{$+$ friend bool operator $<=$ (const Point \& p1, const Point \& p2)} \\
    Function for determining whether all coordinates of a \class{Point}
    are less than or equal to those of another \class{Point}.

    \functionbf{$+$ friend void sort (Point \& p1, Point \& p2)} \\
    Convenience function for putting the lowest coordinates of two
    \class{Point}s in the first \class{Point}, and the highest in the
    second.  On exit, it is guaranteed that \code{p1} $\le$ \code{p2}.


%-----------------------------------------------------------------------
\subsubsection{\class{Point} Data}
%-----------------------------------------------------------------------

    \functionbf{$-$ Coordinate Point::a\_[3]}  \\ \class{Point} coordinates.

%------------------------------------------------------------------------
\subsection{\class{Grid} Class} \label{ss:grid}
%------------------------------------------------------------------------

%**********************************************************************
   \umlGrid
%**********************************************************************

   The \class{Grid} class is for attaching to an application's grid;
   it provides a simplified application interface to application
   grids.

   Note that \amrSolve\ \class{Grid} objects do not have access to
   grid data, even though application grid objects do.  That is the
   responsibility of the \class{Vector} class (\S\ref{ss:vector}), which
   uses the \class{VectorGrid} class to map \class{Grid}s to their data.

   \class{Grid}s also do not inherently have interconnectivity---only
   when they are part of a \class{Level} or \class{Hierarchy} are
   the neighbor and parent/child relationships defined, respectively.
   These relationships are defined in the \class{Level} and \class{Hierarchy}
   objects.

   The \class{Grid} class provides grid-related operations, such as returning
   the size of the grid, and the location of the grid corners.


%-----------------------------------------------------------------------
\subsubsection{\class{Grid} Functions}
%-----------------------------------------------------------------------

    \functionbf{$+$ Grid::Grid ()} \\
    Constructor for the \class{Grid} class.

    \functionbf{$+$ Grid::Grid (const application::grid \&)} \\
    Constructor for creating a \class{Grid} that is attached
    to an application grid.

    \functionit{$+$ void Grid::attach (const application::grid \&)} \\
    Function for attaching to an application's grid.  Called by
    \class{Level} objects.

    \functionit{$+$ void Grid::detach ()} \\
    Function for detaching from an application's grid.  Called by
    \class{Level} objects.

    \functionbf{$+$ void Grid::zones (int \&nu0,int \&nu1,int \&nu2) const} \\
    Return the number of zones along each axis.

    \functionbf{$+$ void Grid::range (Point \& p1, Point \& p2) const} \\ 
    Returns the coordinates of ``lowest'' and ``highest'' corner vertices.

    \functionbf{$+$ bool areNeighbors (const Grid \&g1, const Grid \&g2)} \\ 
    Returns true iff the \class{Grid}s $g1$ and $g2$ are neighbors.
    
    \functionbf{$+$ static const application::grid *Grid::map (const Grid *)} \\
    Returns the application grid that the given \class{Grid} is attached to.

    \functionbf{$+$ static const Grid *Grid::invmap (const
    application::grid *)} \\ 
    Returns the \class{Grid} attached to the given application grid.

    \functionbf{$+$ int Grid::numNeighbors() const} \\
    Returns the number of neighbors of the \class{Grid}.

    \functionbf{$+$ int Grid::numChildren() const} \\
    Returns the number of children of the \class{Grid}.

    \functionbf{$+$ const Grid * Grid::neighbor (int) const} \\
    Returns the $i$th neighbor of the \class{Grid}.

    \functionbf{$+$ const Grid * Grid::child (int) const} \\
    Returns the $i$th child of the \class{Grid}.

    \functionbf{$+$ const Grid * Grid::parent () const} \\
    Returns the parent of the \class{Grid}.

    \functionbf{$+$ const Level \& void Grid::level () const} \\
    Returns the \class{Level} that contains the \class{Grid}.

    \functionbf{$+$ const Hierarchy \& void Grid::hierarchy () const} \\
    Returns the \class{Hierarchy} that contains the \class{Grid}.

    \functionbf{$+$ void Grid::summary () const} \\
    Prints information about this \class{Grid} to \code{stdout} in
    a human-readable format.


%-----------------------------------------------------------------------
\subsubsection{\class{Grid} Data}
%-----------------------------------------------------------------------

%    -------------------------------------------------------------------

    \functionbf{$-$ enzo::grid Grid::application\_grid\_} \\ A pointer
    to the \enzo\ \class{grid} that this \class{Grid} class refers to.

    \functionbf{$-$ static std::map $<$const application::grid *,
    const Grid *$>$ Grid::invmap\_} \\  Mapping of application grids back to
    their \amrSolve\ \class{Grid}.
     
    \functionbf{$-$ short Grid::n\_[3]} \\ The grid size along
    each axis.

    \functionbf{$-$ Point Grid::lp\_ , Grid::hp\_} \\ The low-\class{Point} and
     high-\class{Point} defining the corner vertices of the \class{Grid}.
     \code{lp\_} $\le$ \code{hp\_}.

    \functionbf{$-$ Level * Grid::plevel\_} \\ The \class{Level}, if
     known, containing the \class{Grid}.

%------------------------------------------------------------------------
\subsection{\class{Level} Class} \label{ss:level}
%------------------------------------------------------------------------

   The \class{Level} class us used for encapsulating all \class{Grid}s
   in a level of the hierarchy; that is, all \class{Grid}s of a fixed
   spacial resolution.

%**********************************************************************
   \umlLevel
%**********************************************************************

   \class{Grid}s are \code{friend}s to \class{Level}s, since only
   a \class{Level} can call a \class{Grid}'s \code{attach()} and
   \code{detach()} private member functions.

%-----------------------------------------------------------------------
\subsubsection{\class{Level} Functions}
%-----------------------------------------------------------------------


    \functionbf{$+$ Level::Level ()} \\
    Constructor for the \class{Level} class.

    \functionit{$+$ Level::Level ([application::Level])} \\
    Constructor for attaching to an application's level.

    \functionit{$+$ void Level::attach ([application::Level])} \\
    Attaches to an application's level.

    \functionit{$+$ void Level::detach ()} \\
    Detaches the application's level.

    \functionbf{$+$ int Level::numGrids () const} \\
    Number of \class{Grid}s in the \class{Level}.

    \functionbf{$+$ Grid * Level::grid (int) const} \\ Returns 
    a pointer to \class{Grid} $0 \le i < n_l$ of a \class{Level} $l$, where
    $n_l$ is the number of \class{Grids} in $l$.  If $i <0 $ or $i \ge
    n_l$, then the \code{null} pointer is returned.

    \functionbf{$+$ const Scalar Level::volume () const} \\
    Returns the volume covered by the \class{Grid}s in the \class{Level}.

    \functionbf{$+$ Scalar \& Level::covering () const} \\ Returns the
    covering fraction of grids in the level compared to grids in the
    next-coarser level.  The covering fraction for the coarsest
    \class{Level} is defined as 0.

    \functionbf{$+$ Level * Level::finer () const} \\
    Return a pointer to the next-finer \class{Level} in the \class{Hierarchy}.

    \functionbf{$+$ Level * Level::coarser () const} \\
    Return a pointer to the next-coarser \class{Level} in the \class{Hierarchy}.

    \functionbf{$+$ const Hierarchy \& void Level::hierarchy () const} \\
    Returns the \class{Hierarchy} that contains the \class{Level}.

    \functionbf{$-$ void Level::assertNeighbors (const Grid \& g1, const Grid \& g2) const} \\
    Make \class{Grid}s \code{g1} and \code{g2} neighbors in the \class{Level}.
    Called from \code{attach()}.

%-----------------------------------------------------------------------
\subsubsection{\class{Level} Data}
%-----------------------------------------------------------------------

%-----------------------------------------------------------------------

%    -------------------------------------------------------------------
    \functionbf{$-$ std::vector $<$Grid * $>$ Levl::grids\_} \\
    Vector of pointers to all \class{Grid}s in the \class{Level}.

    \functionbf{$-$ std::multimap $<$const Grid *, const Grid *$>$ Level::neighbors\_} \\
    Multimap of all neighboring \class{Grid}s.  Must be symmetric: if
    \class{Grid} $g_1$ is a \code{neighbor} of $g_2$, then $g_2$ is a
    \code{neighbor} of $g_1$.

    \functionbf{$-$ Hierarchy * Level::phierarchy\_} \\ The
    \class{Hierarchy}, if known, containing the \class{Level}.

%------------------------------------------------------------------------
\subsection{\class{Hierarchy} Class} \label{ss:hierarchy}
%------------------------------------------------------------------------

   The \class{Hierarchy} class attaches to an existing hierarchy in
   the calling application.  \class{Hierarchy}s include functions that
   return various characteristics of the hierarchy, such as number of
   levels, number of grids, and number of grids per level.

   The \class{Hierarchy} class does \textit{not} include iterators for
   traversing grids in a \class{Hierarchy}---that is the responsibility
   of the \class{Iterator} class hierarchy.

%**********************************************************************
   \umlHierarchy
%**********************************************************************

%-----------------------------------------------------------------------
\subsubsection{\class{Hierarchy} Functions}
%-----------------------------------------------------------------------

    \functionbf{$+$ Hierarchy::Hierarchy ()} \\
    Constructor for the \class{Hierarchy} class.

    \functionit{$+$ Hierarchy::Hierarchy ([application::Hierarchy])} \\
    Constructor for attaching to an application's hierarchy.

    \functionit{$+$ void Hierarchy::attach ([application::Hierarchy])} \\
    Attaches to an application's hierarchy.

    \functionit{$+$ void Hierarchy::detach ()} \\
    Detaches the application's hierarchy.

    \functionbf{$+$ int Hierarchy::numLevels () const} \\
    Number of \class{Level}s in the \class{Hierarchy}.

    \functionbf{$+$ int Hierarchy::numGrids () const} \\
    Number of \class{Grid}s in the \class{Hierarchy}.

    \functionbf{$+$ Level * Hierarchy::level (int) const} \\ Returns a pointer
    to \class{Level} $0 \le i < n_h$ of a \class{Hierarchy} $h$, where
    $n_h$ is the number of \class{Level}s in $h$.  If $i <0 $ or $i \ge
    n_h$, then the \code{null} pointer is returned.

    \functionbf{$-$ void Hierarchy::assertParent (const Grid \& g1, const Grid \& g2) const} \\
    Make \class{Grid} \code{g1} a parent of \code{g2} in the \class{Hierarchy}.
    Called from \code{attach()}.

%-----------------------------------------------------------------------
\subsubsection{\class{Hierarchy} Data}
%-----------------------------------------------------------------------

    \functionbf{$-$ std::vector $<$Level * $>$ Hierarchy::levels\_} \\
    Vector of pointers to all \class{Level}s in the \class{Hierarchy}.

    \functionbf{$-$ std::map $<$const Grid *, const Grid *$>$
    Hierarchy::parents\_} \\ Mapping of all parent \class{Grid}s.
    Must be ``antisymmetric with \code{children\_}'': if $g_1$ is a
    \code{parent} of $g_2$, then $g_2$ is a \code{child} of $g_1$.  .

    \functionbf{$-$ std::multimap $<$const Grid *, const Grid *$>$
    Hierarchy::children\_} \\ Mapping of all child \class{Grid}s.
    Must be ``antisymmetric with \code{parent\_}'': if $g_1$ is a
    \code{child} of $g_2$, then $g_2$ is a \code{parent} of $g_1$.

    \functionbf{$-$ std::map $<$const Level *, const Level *$>$
    Hierarchy::coarser\_} \\ Mapping of \class{Level}s to the
    next-coarser one.

    \functionbf{$-$ std::map $<$const Level *, const Level *$>$
    Hierarchy::finer\_} \\ Mapping of \class{Level}s to the
    next-finer one.

%------------------------------------------------------------------------
\subsection{\class{Iterator} Class} \label{ss:iterator}
%------------------------------------------------------------------------

   \class{Iterator} is the abstract base class for several iterator
   classes.  These iterator classes are used to traverse \class{Grid}s
   in a \class{Level} or \class{Hierarchy}, or \class{Level}s in a
   \class{Hierarchy}, in various ways.  An examples is \class{ItGrids}
   for traversing \class{Grid}s in a \class{Level}.  \class{Iterator}
   classes use the linked lists defined by pointers contained in
   \class{Grid} classes to traverse \class{Grid}s.

%**********************************************************************
   \umlIterator
%**********************************************************************

%-----------------------------------------------------------------------
\subsubsection{\class{Iterator} Subclasses}
%-----------------------------------------------------------------------

    \functionbf{$+$ ItLevels} \\
     Traverses all \class{Level}s in a \class{Hierarchy}.

    \functionbf{$+$ ItGrids} \\
     Traverses all \class{Grid}s in a \class{Level}.

    \functionbf{$+$ ItChildren} \\
     Traverses all child \class{Grids}s of a given \class{Grid}.

    \functionbf{$+$ ItNeighbors} \\
     Traverses all adjacent neighboring \class{Grids}s of a given \class{Grid}.

%-----------------------------------------------------------------------
\subsubsection{\class{Iterator} Functions}
%-----------------------------------------------------------------------

    We describe the \class{Iterator} functions in terms of
    \class{OBJECT}s in a \class{CONTAINER}.  Depending on the actual
    \class{Iterator} (\class{ItGrids}, \class{ItLevels}, etc.),
    \class{OBJECT} may mean \class{Grid} or \class{Level}, and
    \class{CONTAINER} may mean \class{Grid}, \class{Level}, or
    \class{Hierarchy}.
    
    \functionbf{$+$ Iterator::Iterator (const OBJECT \&)} \\
    Constructor for the \class{Iterator} class.

    \functionbf{$+$ OBJECT * Iterator::operator ++ ()} \\
    Return a pointer to the current \class{OBJECT}, advancing to the next \class{OBJECT}

    \functionbf{$+$ OBJECT * Iterator::operator *()} \\
    Return a pointer to the current \class{OBJECT}

    \functionbf{$+$ OBJECT * Iterator::begin ()} \\
    Return the first \class{OBJECT}

    \functionbf{$+$ OBJECT * Iterator::end ()                    } \\
    Return one past the last \class{OBJECT} (NULL)

    \functionbf{$+$ void Iterator::reset ()} \\
    Set the current \class{OBJECT} back to the first \class{OBJECT}


%-----------------------------------------------------------------------
\subsubsection{\class{Iterator} Data}
%-----------------------------------------------------------------------

    \functionbf{$-$ CONTAINER * hierarchy\_} \\
    Pointer to the \class{CONTAINER} that we're traversing..

    \functionbf{$-$ OBJECT * current\_} \\
    \functionbf{$-$ int current\_} \\
    Pointer or index identifying the current \class{OBJECT}.


% %------------------------------------------------------------------------
% \subsection{\class{Field} Class} \label{ss:field}
% %------------------------------------------------------------------------
% 
%    \class{Field}s represent scalar fields on the application's grid
%    hierarchy.  It is similar to the \class{Vector} class, but 
%    operations are different--\class{Field} operations tend to be more
%    physics-oriented, whereas \class{Vector} operations tend to be more
%    linear algebra-related.
% 
% %-----------------------------------------------------------------------
% \subsubsection{\class{Field} Subclasses}
% %-----------------------------------------------------------------------
% 
% %-----------------------------------------------------------------------
% \subsubsection{\class{Field} Functions}
% %-----------------------------------------------------------------------
% 
%     \functionbf{$+$ friend VField SField::grad ()} \\
%     Compute the gradient of the \class{SField}.
% 
% %-----------------------------------------------------------------------
% \subsubsection{\class{Field} Data}
% %-----------------------------------------------------------------------
% 

%------------------------------------------------------------------------
\subsection{\class{Vector} Class} \label{ss:vector}
%------------------------------------------------------------------------

   \class{Vector}s represent individual data fields on the
   application's grid hierarchy.  The \class{Vector} class provides an
   interface to the application data fields.

   \class{Vector}s can attach to existing data fields--they don't
   necessarily make separate copies.  In this case, if the application
   changes values in a data field array, the \class{Vector}'s data
   changes.  

   And if the application deletes the data field, the \class{Vector}'s
   data field disappears.  To allow the \class{Vector} class to be
   aware of possible dangling pointers, \class{Vector}s store
   references to application's data pointers, not copies of the
   pointer values.

   \class{Vector}s can also allocate new data fields; for example, for
   temporary vectors in linear solvers, etc.  Such fields are only
   accessible through the \class{Vector} class functions.

   \class{Vector} data for a particular \class{Grid} is located on one
   processor.  Data for different \class{Grid}s can be on different
   processors.  Different data fields for the same \class{Grid} lie on the
   same processor.

%**********************************************************************
   \umlVector
%**********************************************************************

%-----------------------------------------------------------------------
\subsubsection{\class{VectorAbc} Functions}
%-----------------------------------------------------------------------

    \functionit{$+$ void VectorAbc::VectorAbc ()} \\
    Constructor for the VectorAbc abstract base class.

    \functionit{$+$ void VectorAbc::attach ([application::VectorAbc])} \\
    Attaches to an application's data field.

    \functionit{$+$ void VectorAbc::detach ()} \\
    Detaches an application's data field.

    \functionbf{$+$ friend Scalar dot (VectorAbc,VectorAbc)} \\
    Compute the inner-product $a \leftarrow X^T Y$ of two vectors.

    \functionbf{$+$ friend Scalar zaxpy (VectorAbc,Scalar,VectorAbc,VectorAbc)} \\
    Compute the vector update $Z \leftarrow a X + Y$. 

    \functionbf{$+$ void set (Scalar a)} \\
    Sect vector values to given scalar $X \leftarrow a$.

    \functionbf{$+$ void scale (Scalar a)} \\
    Scale vector by the given value: $X \leftarrow a X$.

    \functionbf{$+$ Vector * clone ()} \\
    Clone the vector $X \leftarrow Y$.

%    -------------------------------------------------------------------
    \functionbf{$+$ void refresh ()} \\
    Refresh vector ghost values

%-----------------------------------------------------------------------
\subsubsection{\class{VectorGrid} Functions}
%-----------------------------------------------------------------------

%    -------------------------------------------------------------------

    \functionbf{$+$ Scalar *vertices (Grid)} \\
    Returns a pointer to the array containing the given grid's data,
    beginning with the first ghost-zone.

    \functionbf{$+$ Scalar *unknowns (Grid)} \\
    Returns a pointer to the first array containing the given grid's
    data, beginning with the first actual data zone.

%-----------------------------------------------------------------------
\subsubsection{\class{VectorGrid} Data}
%-----------------------------------------------------------------------


    \functionbf{$-$ Scalar *VectorGrid::v\_} \\
    Arrays pointing to ``vertices'' (grids including ghost zones)

    \functionbf{$-$ short VectorGrid::nd\_[3]} \\
    Array dimensions; 0 if this is a ghost grid.

    \functionbf{$-$ int VectorGrid::gh\_} \\
    Ghost zone depth $0 \le \code{gh\_} \le 1$.

%-----------------------------------------------------------------------
\subsubsection{\class{VectorLevel} Functions}
%-----------------------------------------------------------------------

%-----------------------------------------------------------------------
\subsubsection{\class{VectorLevel} Data}
%-----------------------------------------------------------------------

%-----------------------------------------------------------------------
\subsubsection{\class{VectorHierarchy} Functions}
%-----------------------------------------------------------------------

%-----------------------------------------------------------------------
\subsubsection{\class{VectorHierarchy} Data}
%-----------------------------------------------------------------------

%------------------------------------------------------------------------
\subsection{\class{Matrix} Class} \label{ss:matrix}
%------------------------------------------------------------------------

   The \class{Matrix} class is an abstract base class for representing
   a matrix on the calling application's grid hierarchy.  Specific
   concrete subclasses include \class{Matrix7c}, \class{Matrix7}, and
   \class{Matrix19}, described in the next section.

%**********************************************************************
   \umlMatrix
%**********************************************************************

%-----------------------------------------------------------------------
\subsubsection{\class{Matrix} Subclasses}
%-----------------------------------------------------------------------

    \functionbf{$+$ Matrix7c} \\
      Matrices based on constant coefficient $7$-point stencils.
      E.g. the Laplace operator.

    \functionbf{$+$ Matrix7} \\
      Matrices based on general $7$-point stencils.

    \functionbf{$+$ Matrix19} \\
      Matrices based on general $19$-point stencils.

  
%-----------------------------------------------------------------------
\subsubsection{\class{Matrix} Functions}
%-----------------------------------------------------------------------

    \functionbf{$+$ friend void matvec (Vector Y, Matrix A, Vector X)
    const} \\ Compute the matrix-vector product $Y \leftarrow A X$

    \functionbf{$+$ friend void residual (Vector R, Vector B, Matrix A,
    VectorX) const} \\ Compute the residual $R \leftarrow B - A X$
%    -------------------------------------------------------------------

%-----------------------------------------------------------------------
\subsubsection{\class{Matrix} Data}
%-----------------------------------------------------------------------

%    -------------------------------------------------------------------
    \functionbf{$-$ Hierarchy * hierarchy\_} \\
    Pointer to the \class{Hierarchy} class

    \functionbf{$-$ std::vector $<$Scalar *$>$ a\_} \\
    List of arrays (or pointers to scalars for \class{Matrix7c}) defining
    the matrix diagonals.

%------------------------------------------------------------------------
\subsection{\class{Stencil} Class} \label{ss:stencil}
%------------------------------------------------------------------------

%-----------------------------------------------------------------------
\subsubsection{\class{Stencil} Functions}
%-----------------------------------------------------------------------
%-----------------------------------------------------------------------
\subsubsection{\class{Stencil} Data}
%-----------------------------------------------------------------------

%------------------------------------------------------------------------
\subsection{\class{Discret} Class} \label{ss:discret}
%------------------------------------------------------------------------

%-----------------------------------------------------------------------
\subsubsection{\class{Discret} Functions}
%-----------------------------------------------------------------------
%-----------------------------------------------------------------------
\subsubsection{\class{Discret} Data}
%-----------------------------------------------------------------------

%------------------------------------------------------------------------
\subsection{\class{Solver} Class} \label{ss:solver}
%------------------------------------------------------------------------

   The \class{Solver} class is an abstract base class for representing linear
   solvers.  Specific concrete subclasses include \class{SolverPCG}, for the
   diagonal-preconditioned CG method, and \class{SolverPBiCGSTAB}, for the
   diagonal-preconditioned BiCG-STAB method.  Other solvers, including
   multigrid, may be added in future versions.

%-----------------------------------------------------------------------
\subsubsection{\class{Solver} Subclasses}
%-----------------------------------------------------------------------

    \functionbf{$+$ SolverPCG} \\
     Block-diagonal preconditioned conjugate gradient method.

%-----------------------------------------------------------------------
\subsubsection{\class{Solver} Functions}
%-----------------------------------------------------------------------

%**********************************************************************
   \umlSolver
%**********************************************************************

% -------------------------------------------------------------------
    \functionbf{$+$ void apply (const Matrix \&A, Vector \&X, const Vector
    \&B)} \\ Solves the linear system $A X = B$.

    \functionbf{$+$ int status ()} \\
    Return the status of the latest solve.

%-----------------------------------------------------------------------
\subsubsection{\class{Solver} Data}
%-----------------------------------------------------------------------

    %-------------------------------------------------------------------
    \functionbf{$-$ Matrix A\_   } \\
    Coefficient matrix

    \functionbf{$-$ Vector x\_   } \\
    Solution vector

    \functionbf{$-$ Vector b\_   } \\
    Right-hand side vector

    \functionbf{$-$ Vector r\_   } \\
    Residual vector

    \functionbf{$-$ Vector w\_   } \\
    Work vector

    \functionbf{$-$ Vector p\_   } \\
    Direction vector

    \functionbf{$-$ Precon precon\_} \\
    The preconditioner

    \functionbf{$-$ Scalar tol\_} \\
    The stopping-criteria tolerance
   

%------------------------------------------------------------------------
\subsection{\class{Precon} Class} \label{ss:precon}
%------------------------------------------------------------------------

   The \class{Precon} class is an abstract base class for preconditioners.
   The main proposed preconditioner is a block-diagonal
   preconditioner, where each block corresponds to the coefficient
   matrix on a single \class{Grid}.

%-----------------------------------------------------------------------
\subsubsection{\class{Precon} Subclasses}
%-----------------------------------------------------------------------

    \functionbf{PreconBlockJacobi} \\
    Block Jacobi preconditioner, where blocks are grids, and each grid
    problem is solved using a local \class{Grid} solve.

%-----------------------------------------------------------------------
\subsubsection{\class{Precon} Functions}
%-----------------------------------------------------------------------

    \functionbf{$+$ void apply (const Matrix \&A, Vector \&X, const Vector \&B)} \\
    Apply the preconditioner to the linear system $AX = B$

%-----------------------------------------------------------------------
\subsubsection{\class{Precon} Data}
%-----------------------------------------------------------------------

    \functionbf{$-$ Vector R\_, P\_, W\_, Z\_, \ldots} \\ Temporary
    \class{Vector}s for storing the residual, direction
    \class{Vector}, work \class{Vector}, preconditioned residual, etc.


%------------------------------------------------------------------------
\subsection{\class{Geomview} Class} \label{ss:geomview}
%------------------------------------------------------------------------

     \functionbf{$+$ void Hierarchy::geomview (const FILE *fpr, bool full =
     true) const} \\ Writes \class{Grid} data to a file in geomview format.
     If \code{bool}$=$\code{true}, only write data, not header and
     trailer (for use in animations, etc.)

     \functionbf{$+$ void Level::geomview (const FILE *fpr, bool full =
     true) const} \\ Writes \class{Grid} data to a file in geomview format.
     If \code{bool}$=$\code{true}, only write data, not header and
     trailer (for use by \code{Hierarchy::geomview()}).

     \functionbf{$+$ void Grid::geomview (const FILE *fpr, bool full =
     true) const} \\ Writes \class{Grid} data to a file in geomview
     format.  If \code{bool}$=$\code{true}, only write data, not header
     and trailer (for use by \code{Level::geomview()} and
     \code{Hierarchy::geomview()}).

%==================================================================
\end{document}
%==================================================================


