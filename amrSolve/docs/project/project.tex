%0       1         2         3         4         5         6         7         8
%2345678901234567890123456789012345678901234567890123456789012345678901234567890

\documentclass[11pt]{article}


% INCLUDE DEVELOPMENT TEXT

\newcommand{\devel}[1]{\textbf{#1}}

% EXCLUDE DEVELOPMENT TEXT

% \newcommand{\devel}[1]{}


%=======================================================================
% Document layout
%=======================================================================

\setlength{\topmargin}{-0.5in}
\setlength{\oddsidemargin}{0.0in}
\setlength{\evensidemargin}{0.0in}
\setlength{\textwidth}{6.5in}
\setlength{\textheight}{9.0in}

%=======================================================================
% Packages
%=======================================================================
\newcounter{figctr}

\newcommand{\FIGURE}[3]{
\noindent
\parbox{\textwidth}{
%\ \\ \hrule \ \\
\begin{center}
#3
\end{center}%
\ \nolinebreak%
\refstepcounter{figctr}%
\begin{center}%
\begin{minipage}{7.0in}
\textbf{Figure \thefigctr}. #1
\end{minipage}
\end{center}
\label{#2}
%\ \\ \hrule \ \\
}}

\usepackage{wasysym}
\usepackage{epsfig}
\usepackage{url}

%=======================================================================
% Commands
%=======================================================================

\newcommand{\cello}{\textsf{Cello}}
\newcommand{\enzo}{\textsf{Enzo}}
\newcommand{\lcaperf}{\textsf{lcaperf}}
\newcommand{\lcatest}{\textsf{lcatest}}

\newcommand{\code}[1]{\textsf{#1}}
% not note because conflicts with beamer
\newcommand{\note}[1]{\devel{\eighthnote\ \textit{#1} \\}} 
\newcommand{\pargraph}[1]{\devel{\P\ \textbf{#1} \\}}

\newcommand{\todo}{\devel{$\circ$}}
\newcommand{\done}{\devel{$\bullet$}}
\newcommand{\useyes}{\textcolor{green}{$\bullet$}}
\newcommand{\usemaybe}{\textcolor{yellow}{$\bullet$}}
\newcommand{\useno}{\textcolor{red}{$\bullet$}}
\newcommand{\halfdone}{\devel{\textcolor{gray}{$\bullet$}}}

\newcommand{\TITLE}[3]{
\title{ {\huge #1}  \\ \vspace{0.1in}
     {\small Document Version: #3} \vspace{-0.1in}
    }
\author{      #2 \\
        Laboratory for Computational Astrophysics\\
        University of California, San Diego}
\maketitle}

%=======================================================================


\begin{document}

%========================================================================
\TITLE{\amrSolve \\ Software Project Management Plan}{James Bordner}{0.1.0}
%========================================================================

%========================================================================
\section{Overview}
%========================================================================

%---------------------------------------------
\subsection{Purpose, scope, and objectives}
%---------------------------------------------

\amrSolve\ is designed to be a robust, accurate, efficient, and flexible
parallel elliptic linear solver library for use in structured AMR
(adaptive mesh refinement) codes in general, and in the AMR cosmology
application \enzo~\cite{OsBr04} in particular.  Within \enzo\ it is
expected to be used as a solver option for computing the gravitational
potential, and also as a solver for linear systems arising in implicit
radiation transport methods.  While \enzo\ is the driving application,
\amrSolve\ is designed to be used with other parallel AMR applications
as well.

Robustness is a concern for solving linear systems that arise in
implicit radiation transport algorithms.  Such linear systems can have
very large discontinuities in the coefficient matrices and
right-hand-side vectors, which can adversely affect the convergence of
many iterative methods.

Accuracy is a concern for any elliptic problem discretized on an AMR
hierarchy.  Solving such systems using a na\"{i}ve domain
decomposition approach (similar to that currently used in \enzo\ for
computing the gravitational potential) can easily result in errors of
size $O(h)$ in the refined mesh patches, instead of the expected and
desired $O(h^2)$ errors theoretically possible.  Also, in addition to
its robustness as mentioned above, discontinuities can affect the
accuracy of a solver; for example, at steep discontinuities the
solution can easily become negative, which can lead to non-physical
results (e.g. negative densities, etc.)

Efficiency is a concern for solving elliptic problems in general.
Elliptic problems are non-local in nature, so solving them is
frequently the limiting factor in the parallel scalability of
multiphysics codes such as \enzo.

Lastly, flexibility is a concern due to the different robustness,
accuracy, and efficiency requirements of different linear systems.
For example, solving for the gravitational potential will tend to
stress efficiency, whereas solving a linear system from a radiation
transport method will tend to stress robustness.  To obtain this
flexibility, \amrSolve\ will have a variety of solvers (Krylov
subspace, multigrid, etc.), each with a variety of components
(preconditioners, multigrid operators, etc.), and multiple matrix
datatypes (locally constant coefficient $7$-point stencils, general
$19$-stencils, etc.)

%---------------------------------------------
\subsubsection{Project Deliverables}
%---------------------------------------------

This project's lifetime is taken to be 18 months, 2004-03-01 to
2005-08-31, with the final delivery date nominally being 2005-09-01.
The following are the deliverables for the \amrSolve\ project:

%
\BeginDESCRIPTION
  \item [\Label{D1:} ] \amrSolve\ API (integrated with \enzo)
  \item [\Label{D2:} ] Test suite
  \item [\Label{D3:} ] User guide 
  \item [\Label{D4:} ] Website 
  \item [\Label{D5:} ] Paper 
  \item [\Label{D6:} ] Slides
\EndDESCRIPTION
   

%========================================================================
\section{Project Organization}
%========================================================================

The IEEE standard for software development will serve as a general
guideline for developing the \amrSolve\ software components.  Since this
is a very small scale software project, adherence to the standard will
be flexible.

Documents will include five types: internal, reports, external, papers, and
slides.  Many documents are short, and content in the earlier internal
organizational documents will be used liberally in later, public ones.
The software itself can be considered a sixth document type.  These
document types are described in
\S\S\ref{sss:internal}---\ref{sss:slides}

%------------------------------------------------------------------------

\subsubsection{Internal documentation} \label{sss:internal}

Internal documents are for aiding software development, and are
revised continually as the project progresses.  Since \amrSolve\ is a
one-person project, its internal documents include only the core
subset of those recommended in the IEEE software development standard.
Adherence to the standard sections will also be flexible.

The primary intended audience of the internal documentation is the
software developer; the secondary intended audience are supervisors and
project reviewers.

Since internal documents are working documents, versioning is
necessary.  Versioning will be performed implicitly using the document
date.  Any document on any specific date can be accessed through the
\amrSolve\ Concurrent Versions System (CVS) repository.  

The \amrSolve\ internal documents are listed below:

\BeginENUMERATE
  \item  \amrSolve\ Project Management Plan (\Label{i-spmp})
  \item  \amrSolve\ Requirements Specifications (\Label{i-srs})
  \item  \amrSolve\ Design Description (\Label{i-sdd})
  \item  \amrSolve\ Test Documentation (\Label{i-std})
\EndENUMERATE

%------------------------------------------------------------------------

\subsubsection{Technical Reports}

Technical reports will be used primarily for organizing accumulated
data to be used in research papers.  Emphasis is on data content, not
presentation.  The primary intended audience is the software
developer; the secondary intended audience are computer scientists and
application developers.

\BeginENUMERATE
  \item  \amrSolve\ Method report      (\Label{r-method})
  \item  \amrSolve\ Convergence report (\Label{r-convergence})
  \item  \amrSolve\ Accuracy report    (\Label{r-accuracy})
  \item  \amrSolve\ Performance report (\Label{r-performance})
\EndENUMERATE

The ``Method report'' is intended to describe the \amrSolve\ project
as a whole.  The ``Convergence report,'' ``Accuracy report'', and
``Performance report'' will respectively emphasize the robustness,
accuracy, and performance characteristics of linear solvers in
\amrSolve.

%------------------------------------------------------------------------

\subsubsection{External Documentation}

External documentation is for communicating with \amrSolve\ users and
prospective users.  The primary intended audience is parallel
application programmers who intend to use \amrSolve\ with their
applications.  The user guide will evolve, so will be versioned.
Versioning will be done implicitly using the document date until the
initial public release of \amrSolve.

\BeginENUMERATE
  \item  \amrSolve\ User's Guide      (\Label{e-user}) (\Label{D3})
  \item  \amrSolve\ Website           (\Label{e-www}) (\Label{D4})
\EndENUMERATE

The website will be at \textsf{http://cosmos.ucsd.edu/amrSolve}.

%------------------------------------------------------------------------

\subsubsection{Research Papers}

Research papers are intended to be published in journals or conference
proceedings.  Content will be taken mostly from internal and external
documentation and technical reports.  The primary intended audience is
parallel application developers who are prospective users of
\amrSolve, and scientists with professional interests in
parallel solvers for linear systems on AMR hierarchies.

\BeginENUMERATE
  \item  \amrSolve\ Paper    (\Label{p-amrsolve}) (\Label{D5})
\EndENUMERATE

%------------------------------------------------------------------------

\subsubsection{Slides} \label{sss:slides}

Slides are intended to be presented at conferences and seminars.  The
primary intended audience is parallel application developers who are
prospective users of \amrSolve, and scientists with professional
interests in parallel solvers for linear systems on AMR hierarchies.

\BeginENUMERATE
  \item  \amrSolve\ Slides   (\Label{s-amrsolve}) (\Label{D6})
\EndENUMERATE

%========================================================================
\section{Tasks and Phases}
%========================================================================

There are five phases: 1.~Startup, 2.~Poisson Solver, 3.~General
Solver, 4.~Data Gathering, and 5.~Final Phase.  Active code
development will be primarily in phases 2 and 3.  Documents will be
prepared concurrently throughout the lifecycle, beginning with the
more internal documents, and finishing with the more public documents.

\BeginDESCRIPTION
  \item [Startup Phase: ] Month 1-2
    \BeginITEMIZE
      \item Finish initial drafts of internal documentation
      \item Review Enzo code and datastructures
      \item Review AMR solver literature
    \EndITEMIZE 
  \item [Poisson Solver Phase:] Months 3--8
    \BeginITEMIZE
      \item Iterate over design, code, integrate, test, debug tasks
      \item Begin user guide
      \item Begin website
      \item Begin test suite
    \EndITEMIZE
  \item [General Solver Phase:] Months 9--12
    \BeginITEMIZE
      \item Iterate over design, code, integrate, test, debug tasks
      \item Finish user guide
      \item Finish website
      \item Finish test suite
    \EndITEMIZE
  \item [Data Gathering Phase:] Months 13--16
    \BeginITEMIZE
      \item Run test suite to get convergence, accuracy, and performance data
      \item Accumulate data in corresponding technical reports
      \item Begin \amrSolve\ paper
      \item Begin \amrSolve\ slides
    \EndITEMIZE
  \item [Final Phase:] Months 17--18
    \BeginITEMIZE
      \item Finish \amrSolve\ paper
      \item Finish \amrSolve\ slides
      \item Tie up loose ends
    \EndITEMIZE
\EndDESCRIPTION

%========================================================================
\section{Milestones}
%========================================================================

%------------------------------------------------------------------------
\subsection{Completed milestones}
%------------------------------------------------------------------------

\BeginDESCRIPTION
\item[2004-02: ] Begin Project Management Plan
\item[2004-02: ] Begin Requirements Specifications
\item[2004-03: ] Begin Design Description
\item[2004-03: ] Begin Test Documentation
\item[2004-03: ] \Label{D4} Begin Website
\item[2004-05: ] Implement \class{Point} class
\item[2004-05: ] Implement \class{Grid} class
\item[2004-06: ] Implement \class{Level} class
\item[2004-06: ] Implement \class{Hierarchy} class
\item[2004-06: ] Implement \class{Iterator} class
\EndDESCRIPTION

%------------------------------------------------------------------------
\subsection{Future milestones}
%------------------------------------------------------------------------

\BeginDESCRIPTION
\item[TBD: ] \Label{D3} Begin User guide draft
\item[TBD: ] \Label{D2} Implement test suite
\item[2004-09: ] Implement \class{Field} class
\item[2004-09: ] Implement \class{Vector} class
\item[2004-09: ] Implement \class{Matrix} class (Poisson solver phase)
\item[2004-10: ] Implement \class{Stencil} class (Poisson solver phase)
\item[2004-10: ] Implement \class{Discret} class (Poisson solver phase)
\item[2004-10: ] Implement \class{Solver} class (Poisson solver phase)

\item[TBD: ] Implement \class{Matrix} class (General solver phase)
\item[TBD: ] Implement \class{Stencil} class (General solver phase)
\item[TBD: ] Implement \class{Discret} class (General solver phase)
\item[TBD: ] Implement \class{Solver} class (General solver phase)
\item[TBD: ] \Label{D1} Complete \amrSolve\ code
\item[TBD: ] Write Method report
\item[TBD: ] Write Convergence report
\item[TBD: ] Write Accuracy report
\item[TBD: ] Write Performance report
\item[TBD: ] \Label{D5} Begin Research paper
\item[TBD: ] \Label{D3} Complete User guide
\item[TBD: ] \Label{D4} Complete website
\item[TBD: ] \Label{D5} Complete Research paper
\item[TBD: ] \Label{D6} Write Slides
\item[2005-09-01: ] Public release
\EndDESCRIPTION

% \item[Week 7 (4/01): ]
% \item[Week 8 (4/08): ] 
% \item[Week \, 9 (4/15): ]
% \item[Week 10 (4/22): ] 
% \item[Week 11 (4/29): ]
% \item[Week 12 (5/06): ] 
% \item[Week 13 (5/13): ] 
% \item[Week 14 (5/20): ] 
% \item[Week 15 (5/27): ] 
% \item[Week 16 (6/30): ] 
% \item[Week 17 (6/10): ] 
% \item[Week 18 (6/17): ] 
% \item[Week 19 (6/24): ] 
% \item[Week 20 (7/01): ] 

%==================================================================
\end{document}
%==================================================================


