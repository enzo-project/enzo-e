%0       1         2         3         4         5         6         7         8
%2345678901234567890123456789012345678901234567890123456789012345678901234567890

\documentclass[11pt]{article}


% INCLUDE DEVELOPMENT TEXT

\newcommand{\devel}[1]{\textbf{#1}}

% EXCLUDE DEVELOPMENT TEXT

% \newcommand{\devel}[1]{}


%=======================================================================
% Document layout
%=======================================================================

\setlength{\topmargin}{-0.5in}
\setlength{\oddsidemargin}{0.0in}
\setlength{\evensidemargin}{0.0in}
\setlength{\textwidth}{6.5in}
\setlength{\textheight}{9.0in}

%=======================================================================
% Packages
%=======================================================================
\newcounter{figctr}

\newcommand{\FIGURE}[3]{
\noindent
\parbox{\textwidth}{
%\ \\ \hrule \ \\
\begin{center}
#3
\end{center}%
\ \nolinebreak%
\refstepcounter{figctr}%
\begin{center}%
\begin{minipage}{7.0in}
\textbf{Figure \thefigctr}. #1
\end{minipage}
\end{center}
\label{#2}
%\ \\ \hrule \ \\
}}

\usepackage{wasysym}
\usepackage{epsfig}
\usepackage{url}

%=======================================================================
% Commands
%=======================================================================

\newcommand{\cello}{\textsf{Cello}}
\newcommand{\enzo}{\textsf{Enzo}}
\newcommand{\lcaperf}{\textsf{lcaperf}}
\newcommand{\lcatest}{\textsf{lcatest}}

\newcommand{\code}[1]{\textsf{#1}}
% not note because conflicts with beamer
\newcommand{\note}[1]{\devel{\eighthnote\ \textit{#1} \\}} 
\newcommand{\pargraph}[1]{\devel{\P\ \textbf{#1} \\}}

\newcommand{\todo}{\devel{$\circ$}}
\newcommand{\done}{\devel{$\bullet$}}
\newcommand{\useyes}{\textcolor{green}{$\bullet$}}
\newcommand{\usemaybe}{\textcolor{yellow}{$\bullet$}}
\newcommand{\useno}{\textcolor{red}{$\bullet$}}
\newcommand{\halfdone}{\devel{\textcolor{gray}{$\bullet$}}}

\newcommand{\TITLE}[3]{
\title{ {\huge #1}  \\ \vspace{0.1in}
     {\small Document Version: #3} \vspace{-0.1in}
    }
\author{      #2 \\
        Laboratory for Computational Astrophysics\\
        University of California, San Diego}
\maketitle}

%=======================================================================


\begin{document}

%========================================================================
\TITLE{\amrSolve \\ Software Project Management Plan}{James Bordner}{1.2}
%========================================================================

%========================================================================
\section{Modification History}
%========================================================================

\BeginDESCRIPTION
\item[1.2 (2004-03-16):] Revised text, leaving content unchanged
\item[1.2 (2004-03-16):] Added Modification History section
\EndDESCRIPTION


%========================================================================
\section{Overview}
%========================================================================

%---------------------------------------------
\subsection{Purpose, scope, and objectives}
%---------------------------------------------

\amrSolve\ is designed to be a small, robust, accurate, scalable, and efficient
parallel elliptic linear solver for use in structured AMR codes in
general, and \enzo\ in particular.  It should be especially suitable
for computing gravitational potential fields, and as a linear solver
for implicit radiation transport methods.

%---------------------------------------------
\subsubsection{Project Deliverables}
%---------------------------------------------

This project's lifetime is taken to be 2004-02-12 to 2004-06-30, with
a delivery date of 2004-07-01.  Below are the deliverables from the
\amrSolve\ project.
%
\BeginDESCRIPTION
  \item [\Label{D1:} ] \amrSolve\ code integrated with \enzo
  \item [\Label{D2:} ] Test suite
  \item [\Label{D3:} ] User guide 
  \item [\Label{D4:} ] Website 
  \item [\Label{D5:} ] Paper 
  \item [\Label{D6:} ] Slides
\EndDESCRIPTION
   

%========================================================================
\section{Project Organization}
%========================================================================

The IEEE standard for software development will serve as a general
guideline for developing the \amrSolve\ code.  Since this is a 
small-scale software project, adherence to the standard will be
flexible.

%------------------------------------------------------------------------
\subsection{Documents}
%------------------------------------------------------------------------

Documents will include five types: internal, reports, external,
papers, and slides. Most are relatively short, and content in the
earlier organizational documents will be liberally used in later
public ones.  The code itself can be considered a sixth document type.
These five document types are described in
\S\S\ref{sss:internal}---\ref{sss:slides}

%------------------------------------------------------------------------

\subsubsection{Internal documentation} \label{sss:internal}

Internal documentation is for aiding software development.  The
\amrSolve\ internal documents include only the core subset of IEEE
software development standard documents.  Adherence to the standard
sections will also be flexible.  The intended audience is the software
developer.  Documents will evolve, so will be versioned.

\BeginENUMERATE
  \item  \amrSolve\ Project Management Plan (\Label{i-spmp})
  \item  \amrSolve\ Requirements Specifications (\Label{i-srs})
  \item  \amrSolve\ Design Description (\Label{i-sdd})
  \item  \amrSolve\ Test Documentation (\Label{i-std})
\EndENUMERATE

%------------------------------------------------------------------------

\subsubsection{Technical Reports}

Technical reports will be used primarily for organizing accumulated
data to be used in research papers.  Emphasis is on data content, not
presentation.  The intended audience is the software developer, but
may be made publicly available as LCA technical reports.  Documents
will evolve, so will be versioned.

\BeginENUMERATE
  \item  \amrSolve\ Method report      (\Label{r-method})
  \item  \amrSolve\ Convergence report (\Label{r-convergence})
  \item  \amrSolve\ Accuracy report    (\Label{r-accuracy})
  \item  \amrSolve\ Performance report (\Label{r-performance})
\EndENUMERATE

%------------------------------------------------------------------------

\subsubsection{External Documentation}

External documentation is for communicating with \amrSolve\ users and
prospective users.  The intended audience is the public.  The user
guide will evolve, so will be versioned.

\BeginENUMERATE
  \item  \amrSolve\ User's Guide      (\Label{e-user}) (\Label{D3})
  \item  \amrSolve\ Website           (\Label{e-www}) (\Label{D4})
\EndENUMERATE

The website will be at \textsf{http://cosmos.ucsd.edu/amrSolve}.

%------------------------------------------------------------------------

\subsubsection{Research Papers}

Research papers are intended to be published in journals or conference
proceedings.  Content will be taken mostly from technical reports.
The intended audience is the public.

\BeginENUMERATE
  \item  \amrSolve\ Paper    (\Label{p-amrsolve}) (\Label{D5})
\EndENUMERATE

%------------------------------------------------------------------------

\subsubsection{Slides} \label{sss:slides}

The time-length will be roughly one hour, so that a shorter talk can
be given using a subset of the slides.  The intended audience for
slides is the public.

\BeginENUMERATE
  \item  \amrSolve\ Slides   (\Label{s-amrsolve}) (\Label{D6})
\EndENUMERATE

%========================================================================
\section{Tasks and Phases}
%========================================================================

There are five phases: 1.~Startup, 2.~Poisson Solver, 3.~General
Solver, 4.~Data Gathering, and 5.~Final Phase.  Active code
development will be primarily in phases 2 and 3.  Documents will be
prepared concurrently throughout the lifecycle, beginning with the
more internal documents, and finishing with the more public documents.

\BeginDESCRIPTION
  \item [Startup Phase: ] Weeks 1--4
    \BeginITEMIZE
      \item Finish initial drafts of internal documentation
      \item Review Enzo code and datastructures
      \item Review AMR solver literature
    \EndITEMIZE 
  \item [Poisson Solver Phase:] Weeks 5-8
    \BeginITEMIZE
      \item Iterate over design, code, integrate, test, debug tasks
      \item Begin user guide
      \item Begin website
      \item Begin test suite
    \EndITEMIZE
  \item [General Solver Phase:] Weeks 9--12
    \BeginITEMIZE
      \item Iterate over design, code, integrate, test, debug tasks
      \item Finish user guide
      \item Finish website
      \item Finish test suite
    \EndITEMIZE
  \item [Data Gathering Phase:] Weeks 13--16
    \BeginITEMIZE
      \item Run test suite to get convergence, accuracy, and performance data
      \item Accumulate data in corresponding ``r-'' reports
      \item Begin \amrSolve\ paper
      \item Begin \amrSolve\ slides
    \EndITEMIZE
  \item [Final Phase:] Weeks 17--20
    \BeginITEMIZE
      \item Finish \amrSolve\ paper
      \item Finish \amrSolve\ slides
      \item Tie up loose ends
    \EndITEMIZE
\EndDESCRIPTION



%========================================================================
\section{Weekly Milestones}
%========================================================================

\subsection{Startup phase}

\BeginDESCRIPTION
\item[Week 1 (2/19): ] Project Management Plan Draft
\item[Week 2 (2/26): ] Requirements Specifications Draft
\item[Week 3 (3/04): ] Design Description Draft
\item[Week 4 (3/11): ] Test Documentation Draft
\EndDESCRIPTION

\subsection{Poisson Solver Phase}

\BeginDESCRIPTION
\item[Week 5 (3/18): ] 
\item[Week 6 (3/25): ] \Label{D4} Website draft
\item[Week 7 (4/01): ]
\item[Week 8 (4/08): ] \Label{D3} User guide draft
\EndDESCRIPTION

\subsection{General Solver Phase}

\BeginDESCRIPTION
\item[Week \, 9 (4/15): ]
\item[Week 10 (4/22): ] \Label{D2} Test suite draft
\item[Week 11 (4/29): ]
\item[Week 12 (5/06): ] \Label{D1} \amrSolve\ code
\EndDESCRIPTION

\subsection{Data Gathering Phase}

\BeginDESCRIPTION
\item[Week 13 (5/13): ] Method report
\item[Week 14 (5/20): ] Convergence report
\item[Week 15 (5/27): ] Accuracy report
\item[Week 16 (6/30): ] Performance report
\EndDESCRIPTION

\subsection{Final Phase}

\BeginDESCRIPTION
\item[Week 17 (6/10): ] 
\item[Week 18 (6/17): ] \Label{D5} Research paper
\item[Week 19 (6/24): ] \Label{D6} Slides
\item[Week 20 (7/01): ] Public release
\EndDESCRIPTION

%==================================================================
\end{document}
%==================================================================


